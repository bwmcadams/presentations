\documentclass[xcolor=dvipsnames]{beamer}

% Copyright 2010, 2011 by Brendan W. McAdams <brendan@10gen.com>
%
% You may redistribute the content of this presentation for your own needs 
% provided you give credit to its author. In other words: "Don't be a dick."
% % Latex code based upon the "Generic Presentation 15-45 minutes" template from Beamer.

\mode<presentation>
{
  % \usetheme{Dresden}      
  \usetheme{Madrid}
  \setbeamercolor{structure}{fg=OliveGreen!125!black} 
        
  % \usecolortheme{dolphin}
  \setbeamercovered{transparent}
}


\usepackage{hyperref}
\usepackage{xcolor}
\usepackage{listings}
\usepackage{ulem}

\hypersetup{backref,  
pdfpagemode=FullScreen,  
colorlinks=false}

 % "define" Scala
\lstdefinelanguage{Scala}{
   morekeywords={abstract,case,catch,class,def,%
     do,else,extends,false,final,finally,%
     for,forSome,if,implicit,import,lazy,%
     match,mixin,%
     new,null,object,override,package,%
     private,protected,requires,return,sealed,%
     super,this,throw,trait,true,try,%
     type,val,var,while,with,yield},
   otherkeywords={=>,<-,<\%,<:,>:,\#,@},
   sensitive=true,
   morecomment=[l]{//},
   morecomment=[n]{/*}{*/},
   morestring=[b]",
   morestring=[b]',
   morestring=[b]"""
 }


\lstdefinelanguage{JavaScript}{
    keywords={typeof, new, true, false, catch, function, return, null, catch, switch, var, if, in, while, do, else, case, break}
    keywordstyle=\color{blue}\bfseries,
    ndkeywords={class, export, boolean, throw, implements, import, this},
    ndkeywordstyle=\color{darkgray}\bfseries,
    identifierstyle=\color{black},
    sensitive=false,
    comment=[l]{//},
    morecomment=[s]{/*}{*/},
    commentstyle=\color{purple}\ttfamily,
    stringstyle=\color{red}\ttfamily,
    morestring=[b]',
    morestring=[b]"
}
\usepackage{caption}
\DeclareCaptionFont{white}{\color{white}}
\DeclareCaptionFormat{listing}{\colorbox{black}{\parbox{\textwidth}{#1#2#3}}}
\captionsetup[lstlisting]{format=listing,labelfont=white,textfont=white}


\usepackage{color}
\definecolor{dkgreen}{rgb}{0,0.6,0}
\definecolor{gray}{rgb}{0.5,0.5,0.5}
\definecolor{mauve}{rgb}{0.58,0,0.82}
\definecolor{lightgray}{rgb}{.9,.9,.9}
\definecolor{darkgray}{rgb}{.4,.4,.4}
\definecolor{purple}{rgb}{0.65, 0.12, 0.82}

% Default settings for code listings
\lstset{
  frame=tb,%
  language=Scala,%
  aboveskip=1.5mm,%
  belowskip=1.5mm,%
  xleftmargin=1mm,%
  xrightmargin=1mm,%
  showstringspaces=false,%
  keepspaces=true,%
  columns=[c]fixed,%
  basicstyle={\tiny\ttfamily},%
  escapechar=¤,%
  numbers=none,%
  numberstyle=\tiny\color{yellow},%
  keywordstyle=\color{blue},%
  commentstyle=\color{dkgreen},%
  stringstyle=\color{mauve},%
  frame=single,%
  breaklines=true,%
  breakatwhitespace=true,%
  tabsize=4,%
  firstnumber=0,%
  numbersep=1.5mm,%
  numberstyle=\tiny%
}

% \lstdefinestyle{floating}{%
%     xleftmargin=10pt,%
%     xrightmargin=5pt,%
%     aboveskip=4mm,%
%     belowskip=4mm,%
%     fontadjust=true,%
%     columns=[c]flexible,%
%     keepspaces=true,%
%     basewidth={0.5em, 0.425em},%
%     tabsize=2,%
%     basicstyle=\renewcommand{\baselinestretch}{0.95}\ttfamily,%
%     commentstyle=\rm,%
%     keywordstyle=\bfseries,%
%     mathescape=true,%
%     captionpos=b,%
%     framerule=0.3pt,%
%     firstnumber=0,%
%     numbersep=1.5mm,%
%     numberstyle=\tiny,%
%     float=tbp,%
%     frame=tblr,%
%     framesep=5pt,%
%     framexleftmargin=3pt,%
%     abovecaptionskip=\smallskipamount,%
%     belowcaptionskip=\smallskipamount,%
% } % to define: caption, label

\newcommand{\code}[1]{%
    \lstinline[%keywordstyle=,%
               flexiblecolumns=true,%
               basicstyle=\ttfamily]_#1_}

\newenvironment{itemizeframe}
               {\begin{frame}\startitemizeframe} 
               {\stopitemizeframe\end{frame}}
              
\newenvironment{codeframe}
                {\begin{frame}[allowframebreaks,allowdisplaybreaks]}
                {\end{frame}}
                
\newenvironment{itemizecodeframe}
              {\begin{frame}[allowframebreaks,allowdisplaybreaks]
              \startitemizeframe} 
              {\stopitemizeframe\end{frame}}

\newcommand\startitemizeframe{\begin{itemize}} \newcommand\stopitemizeframe{\end{itemize}}

\usepackage[english]{babel}
% or whatever

\usepackage[latin1]{inputenc}
% or whatever

\usepackage{times}
\usepackage[T1]{fontenc}
% Or whatever. Note that the encoding and the font should match. If T1
% does not look nice, try deleting the line with the fontenc.


\title{Hammersmith: Netty, Scala, and MongoDB} % (optional, use only with long paper titles)

\institute[10gen, Inc.]{10gen, Inc.}

%\subtitle
%{Presentation Subtitle} % (optional)

\author[B.W. McAdams]{Brendan W. McAdams}
% - Use the \inst{?} command only if the authors have different
%   affiliation.

\date{NY Scala Enthusiasts}

\subject{Hammersmith}
% This is only inserted into the PDF information catalog. Can be left
% out. 


\logo{\includegraphics[width=2.25cm]{images/mongodb-logo}}


% Delete this, if you do not want the table of contents to pop up at
% the beginning of each subsection:
% \AtBeginSubsection[]
% {
%   \begin{frame}<beamer>
%   \frametitle{Outline}
%     \tableofcontents[currentsection,currentsubsection,currentsubsubsection]
%   \end{frame}
% }


% If you wish to uncover everything in a step-wise fashion, uncomment
% the following command: 

% \beamerdefaultoverlayspecification{<+->}


\begin{document}

\begin{frame}
  \titlepage
\end{frame}

\section{Introducing Hammersmith}

\begin{itemizeframe}
    \frametitle{What is Hammersmith?}
    \framesubtitle{Overview}
    \item New MongoDB Driver for Scala 
    \item Distillation of Lessons Learned in 18 Months of Casbah (Spiritual "cousin" rather than successor)
    \item Pure Scala (With a little bit of Java at the very bottom for BSON)
    \item Focused more on \textit{frameworks} than \textit{userspace}
    \item Purely Asynchronous and Non-Blocking Network I/O
\end{itemizeframe}

\begin{itemizeframe}
    \frametitle{What is Hammersmith?}
    \framesubtitle{What Problems Does It Solve/Explore?}
    \item Learn BSON and the MongoDB Wire Protocol
    \item Architectural Evolutions from Java Driver (and Casbah Lessons)
        \begin{itemize} 
            \item Better Connection Pools and Cleaner Type tree for Connections (Direct, Replica Set, Master/Slave)
            \item Faster, cleaner and more extensible pluggability for custom serialization of business objects <-> BSON
        \end{itemize}
    \item Asynchronous Networking
        \begin{itemize}
            \item Integrate more appropriately with purely asynchronous frameworks like "BlueEyes" \url{https://github.com/jdegoes/blueeyes}
            \item Get away from any synchronization, threading and blocking which can limit the scalability ceiling of working with MongoDB
        \end{itemize}
    \item \sout{World Domination!}
\end{itemizeframe}

\section{Hammersmith in Code}

\begin{itemizecodeframe}
    \frametitle{Speaking to MongoDB Asynchronously}
    \item In an asynchronous framework, we must be careful to never block 
    \item The problem of course, is how do you manage a truly synchronous operation like talking to a database?
    \pagebreak
    \item Anonymous Function Callbacks and Dispatching by Request ID does the trick nicely!
        \lstinputlisting{code/callback_ops.scala}
\end{itemizecodeframe}


\begin{itemizecodeframe}
    \frametitle{Speaking to MongoDB Asynchronously}
    \framesubtitle{Cursors}
    \item Handling a findOne or command, which return a single Document is not difficult.
    \item Because of batching, Cursors (Which MongoDB uses where number of matching docs > 1) are difficult to do asynchronously without blocking
    \item Standard iteration (say, calling "next()") has in essence, two return values:
        \begin{itemize}
            \item Some(value) ... A valid value was found in the iterator and returned
            \item None ... No value was found, this iterator is empty and done
        \end{itemize}
    \pagebreak
    \item A cursor (MongoDB's are much like any other database's) returns documents in batches to save memory on the client.  They really have \textit{three} states:
        \begin{itemize}
            \item Entry(value) ... A valid value was found in the *CLIENTS LOCAL BATCH* and returned 
            \item Empty ... The client's local batch is empty but there appear to be more results on the server
            \item EOF ... The client and server have exhausted their results and nothing more is to be had.
        \end{itemize}
    \item Solution: Haskell's Iteratee Pattern (also available in scalaz).  Suggested by @jdegoes (thanks!)
    \item The three states are represented as an "IterState" and the cursor iteration is controlled by "IterCmd" issued by a user function in response to "IterState".
        \lstinputlisting{code/iteree.scala}
\end{itemizecodeframe}

%\begin{itemizeframe}
    %\frametitle{Pimpin' Your Library}
    %\item Implicit methods allow for the the ``Pimp My Library Pattern''
    %\item Coined by Martin Odersky in a 2006 Blog post.  Similar to C\# extension methods, Ruby modules.
    %\item Uses implicit conversions to tack on new methods at runtime.
    %\item Either return a new ``Rich\_'' or anonymous class\ldots
        %\lstinputlisting{code/pimp_sample1.scala}

%\end{itemizeframe}

\begin{itemizecodeframe}
    \frametitle{What's Next?}
    \item Still fleshing out the BSON layer, but focused on being able to plug completely custom Business Objects in at a low level
    \item Working on Connection Pools (with commons-pool)
    \item Aiming to Release an Alpha in the next 2-3 weeks
    \item The goal here is to be partly community driven: Making this fit into the frameworks in a way that benefits the Scala + MongoDB user bases. I want your contributions and thoughts!
\end{itemizecodeframe}

\subsection{Closing}
\begin{itemizeframe}
    \frametitle{Questions?}
        \item Twitter: {\bf @rit} | mongodb: {\bf @mongodb} | 10gen: {\bf @10gen }
        \item email: {\bf brendan@10gen.com}	
    
    \par
    \item Upcoming MongoDB Events
    \begin{itemize}
        \item Mongo Philly (April 26, 2011) - \url{http://bit.ly/mongophilly}        
        \item Mongo NY (June 7, 2011) - \url{http://www.10gen.com/conferences/mongonyc2011}
    \end{itemize}
    \par
    \item 10gen is hiring!  We need smart engineers (C++) in both NY and the Bay Area (Redwood Shores): \url{http://10gen.com/jobs}

\end{itemizeframe}



\end{document}
