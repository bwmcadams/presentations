\documentclass[xcolor=dvipsnames]{beamer}

% Copyright 2010 by Brendan W. McAdams <brendan@10gen.com>
%
% You may redistribute the content of this presentation for your own needs 
% provided you give credit to its author. In other words: "Don't be a dick."
% % Latex code based upon the "Generic Presentation 15-45 minutes" template from Beamer.


\mode<presentation>
{
  % \usetheme{Dresden}      
  \usetheme{Madrid}
  \setbeamercolor{structure}{fg=OliveGreen!125!black} 
        
  % \usecolortheme{dolphin}
  \setbeamercovered{transparent}
}


\usepackage{hyperref}
\usepackage{xcolor}
\usepackage{listings}

\hypersetup{backref,  
pdfpagemode=FullScreen,  
colorlinks=false}

 % "define" Scala
\lstdefinelanguage{Scala}{
   morekeywords={abstract,case,catch,class,def,%
     do,else,extends,false,final,finally,%
     for,forSome,if,implicit,import,lazy,%
     match,mixin,%
     new,null,object,override,package,%
     private,protected,requires,return,sealed,%
     super,this,throw,trait,true,try,%
     type,val,var,while,with,yield},
   otherkeywords={=>,<-,<\%,<:,>:,\#,@},
   sensitive=true,
   morecomment=[l]{//},
   morecomment=[n]{/*}{*/},
   morestring=[b]",
   morestring=[b]',
   morestring=[b]"""
 }


\lstdefinelanguage{JavaScript}{
    keywords={typeof, new, true, false, catch, function, return, null, catch, switch, var, if, in, while, do, else, case, break}
    keywordstyle=\color{blue}\bfseries,
    ndkeywords={class, export, boolean, throw, implements, import, this},
    ndkeywordstyle=\color{darkgray}\bfseries,
    identifierstyle=\color{black},
    sensitive=false,
    comment=[l]{//},
    morecomment=[s]{/*}{*/},
    commentstyle=\color{purple}\ttfamily,
    stringstyle=\color{red}\ttfamily,
    morestring=[b]',
    morestring=[b]"
}
\usepackage{caption}
\DeclareCaptionFont{white}{\color{white}}
\DeclareCaptionFormat{listing}{\colorbox{black}{\parbox{\textwidth}{#1#2#3}}}
\captionsetup[lstlisting]{format=listing,labelfont=white,textfont=white}


\usepackage{color}
\definecolor{dkgreen}{rgb}{0,0.6,0}
\definecolor{gray}{rgb}{0.5,0.5,0.5}
\definecolor{mauve}{rgb}{0.58,0,0.82}
\definecolor{lightgray}{rgb}{.9,.9,.9}
\definecolor{darkgray}{rgb}{.4,.4,.4}
\definecolor{purple}{rgb}{0.65, 0.12, 0.82}

% Default settings for code listings
\lstset{
  frame=tb,%
  language=Scala,%
  aboveskip=1.5mm,%
  belowskip=3.5mm,%
  xleftmargin=1mm,%
  xrightmargin=2.5mm,%
  showstringspaces=false,%
  keepspaces=true,%
  columns=[c]fixed,%
  basicstyle={\scriptsize\ttfamily},%
  escapechar=¤,%
  numbers=none,%
  numberstyle=\tiny\color{yellow},%
  keywordstyle=\color{blue},%
  commentstyle=\color{dkgreen},%
  stringstyle=\color{mauve},%
  frame=single,%
  breaklines=true,%
  breakatwhitespace=true,%
  tabsize=4,%
  firstnumber=0,%
  numbersep=1.5mm,%
  numberstyle=\tiny,
  title=\lstname%
}

\newcommand{\code}[1]{%
    \lstinline[%keywordstyle=,%
               flexiblecolumns=true,%
               basicstyle=\ttfamily]_#1_}

\newenvironment{itemizeframe}
               {\begin{frame}\startitemizeframe} 
               {\stopitemizeframe\end{frame}}
              
\newenvironment{codeframe}
                {\begin{frame}[allowframebreaks,allowdisplaybreaks]}
                {\end{frame}}
                
\newenvironment{itemizecodeframe}
              {\begin{frame}[allowframebreaks,allowdisplaybreaks]
              \startitemizeframe} 
              {\stopitemizeframe\end{frame}}

\newcommand\startitemizeframe{\begin{itemize}} \newcommand\stopitemizeframe{\end{itemize}}

\usepackage[english]{babel}
% or whatever

\usepackage[latin1]{inputenc}
% or whatever

\usepackage{times}
\usepackage[T1]{fontenc}
% Or whatever. Note that the encoding and the font should match. If T1
% does not look nice, try deleting the line with the fontenc.


\title{Migrating from Casbah 1.x -> 2.0} % (optional, use only with long paper titles)

\institute[10gen, Inc.]{10gen, Inc.}

%\subtitle
%{Presentation Subtitle} % (optional)

\author[B.W. McAdams]{Brendan W. McAdams}
% - Use the \inst{?} command only if the authors have different
%   affiliation.

\date{Dec 14, 2010}

\subject{Migrating from Casbah 1.x -> 2.0}
% This is only inserted into the PDF information catalog. Can be left
% out. 


\logo{\includegraphics[width=2.25cm]{images/mongodb-logo}}

% If you wish to uncover everything in a step-wise fashion, uncomment
% the following command: 

% \beamerdefaultoverlayspecification{<+->}
 
\AtBeginSubsection[]
{
  \begin{frame}<beamer>
  \frametitle{Outline}
  \tableofcontents[currentsection,currentsubsection,currentsubsubsection]
  \end{frame}
}

\begin{document}

\begin{frame}
  \titlepage
  \begin{center}
  \includegraphics[scale=0.25]{images/powered_mongo.png}
  \end{center}
\end{frame}

\section{Introduction}

\begin{itemizeframe}
    \frametitle{Casbah 2.0}
    \framesubtitle{What Changed?}
    \item<+-> Now a 10gen supported driver
        \begin{itemize}
            {\scriptsize 
            \item Package Change: now `com.mongodb.casbah'
            }
        \end{itemize}
    \item<+-> Sanity Cleanups and Improvements
        \begin{itemize}
            {\scriptsize 
            \item Modularized: Decouples some dependencies, pick and choose needed components
            \item Removed `configgy', replaced with a bare minimum `slf4j' implementation for logging
            \item Type conversions now always loaded when Commons is imported, rather than on Connection creation
            \item Removal of `global' implicit Tuple/Product -> DBObject Conversions
            \item Added Scala 2.8 Collections-style DBList implementation, `MongoDBList'
            }
        \end{itemize}
    \item<+-> Updated to Java Driver 2.3/2.4 w/ full API compatibility
        \begin{itemize}
            {\scriptsize 
            
            \item Introduced Write Concern functionality including execute-around safe mode   
            }         
        \end{itemize}
    \item<+-> Significantly improved Query DSL, including support for all operators
        \begin{itemize}
            {\scriptsize 
            
            \item Introduced Type Class Context Boundaries for DSL to allow user implementation of custom types while still maintaining type safety
            \item Operators now return DBObject instead of Tuples
            }
        \end{itemize}
\end{itemizeframe}

\section{The Details}

\subsection{Modularization and Packaging}

\begin{frame}
    \frametitle{New Packaging}
    As part of the move to a 10gen supported package, Casbah's package has changed. The now defunct (it was rolled into 2.0) 1.1 branch was changing the package as well, and rolled forward.
    {\scriptsize
    \begin{table}[h]
        \caption{Casbah Packaging}
        \centering
        \begin{tabular}{c c c}
            \hline\hline
            Casbah 1.0.x & Casbah 1.1.x & Casbah 2.0+ \\ [0.5ex]
            \hline
            com.novus.casbah.mongodb & com.novus.casbah & com.mongodb.casbah \\ 
            \hline
        \end{tabular}
        \label{table:packaging}
    \end{table}
    }
\end{frame}

\begin{frame}
    \frametitle{Modularization}
    Some users (such as those using a framework like Lift which already provides MongoDB wrappers) wanted access to certain parts of Casbah (e.g. MongoDBObject \& MongoDBList) without importing the whole system. As a result, Casbah has been broken out into several modules which make it easier to pick and choose the features you want.
\end{frame}

\begin{itemizeframe}
    \frametitle{Casbah Modules}
    \framesubtitle{Commons (casbah-commons)}
    \item Module: `casbah-commons' ("Commons") Package: com.mongodb.casbah.commons
        \begin{itemize}
            \item Dependent upon mongo-java-driver, scalaj-collection, scalaj-time, JodaTime, slf4j-api
            \item Provides core Scala bindings via the Conversions API (now autoloaded when Commons is imported) for converting to and from common Scala and Java types transparently
            \item Provides Scala 2.8 Collections compatible wrappers for `DBObject' and `DBList' as `MongoDBObject' and `MongoDBList' respectively
        \end{itemize}
    \item Raw import of just Commons via: \\
        \code{import com.mongodb.casbah.commons.Imports.\_}
\end{itemizeframe}


\begin{itemizeframe}
    \frametitle{Casbah Modules}
    \framesubtitle{Query DSL (casbah-query)}
    \item Module: `casbah-query' ("Query DSL") Package: com.mongodb.casbah.query
        \begin{itemize}
            \item Dependent upon casbah-commons along with any transitive dependencies
            \item Provides the Scala syntax DSL mode for creating MongoDB query objects via `\$ Operators', e.g.:\\
                \code{"foo" \$lt 50 \$gte 12}
            \item Using this gives you only type conversions, DBObject and DBList wrappers and the DSL.
        \end{itemize}
    \item Raw import of just Query DSL (provides Commons automatically) via: \\
        \code{import com.mongodb.casbah.query.Imports.\_}
\end{itemizeframe}

\begin{itemizeframe}
    \frametitle{Casbah Modules}
    \framesubtitle{Core (casbah-core)}
    \item Module: `casbah-core' ("Core") Package: com.mongodb.casbah.core
        \begin{itemize}
            \item Dependent upon casbah-commons and casbah-query along with their dependencies transitively
            \item Provides  Scala bindings to the Java driver for actually communicating with MongoDB e.g. `DB' and `DBCollection' and MapReduce jobs, wraps the Mongo wire format with Scala type support.
            \item Importing Core will give you identical functionality to 1.x
        \end{itemize}
    \item Raw import of Core (provides Commons and QueryDSL automatically) via: \\
        \code{import com.mongodb.casbah.Imports.\_}
\end{itemizeframe}

\begin{itemizeframe}
    \frametitle{Casbah Modules}
    \framesubtitle{GridFS (casbah-gridfs)}
    \item Module: `casbah-gridfs' ("GridFS") Package: com.mongodb.casbah.gridfs
        \begin{itemize}
            \item Dependent upon casbah-core and casbah-commons along with their dependencies transitively
            \item Provides Scala enhanced wrappers to MongoDB's GridFS filesystem.
            \item NOT imported by default, as many people don't use or need it.  You need to explicitly import this if you want GridFS support.
        \end{itemize}
    \item Raw import of GridFS via: \\
        \code{import com.mongodb.casbah.gridfs.Imports.\_}
\end{itemizeframe}

\subsection{Sanity Improvements}
\begin{codeframe}
    \frametitle{Removal of global Product/Tuple Conversions}
    Previously, it was possible with Casbah to cast Tuples to DBObject:\\

    \code{val x: DBObject = ("foo" -> "bar", "x" -> 5, "y" -> 238.1)}
    \\
    This feature was provided by implicit conversions which attempt to target Product which is the base class of all Tuples. Buggy \& unreliable: often targeted the wrong things for conversion (Such as instances of Option[\_]). As Casbah 2.0 includes wrappers for DBObject which follow Scala 2.8's Collection interfaces including builders and constructors, replaced these conversions. Previous syntax is possible by passing the Tuple pairs to MongoDBObject.apply:\\
    \lstinputlisting{code/tuple2.scala}
    \pagebreak
    We also provide a builder pattern which follows Scala 2.8's Map Builder:\\
    \lstinputlisting{code/tuple3.scala}
    \pagebreak
    Finally, any Scala map can still be cast to a DBObject without issue:\\
    \lstinputlisting{code/tuple4.scala}
    \pagebreak
    It is still possible to use Tuples in the Query DSL however, as there is less need for broad implicit conversions to accomplish that functionality.
\end{codeframe}

\subsection{Java Driver Updates}

\begin{itemizeframe}
    \frametitle{Various Improvements to Java Driver Support on Core}
    \item Brought Casbah's Core API in line with Java Driver 2.3
    \item Support for \code{MongoURI}, \code{MongoOptions}, \code{slaveOK} \& \code{WriteConcern}
    \item Support for Replica Sets
    \item Expanded many methods which previously took DBObject to take a View Boundary e.g.:\\
            \code{def foo[A <\% DBObject](arg: A) = \{ /* .. */ \} }
\end{itemizeframe}

\begin{itemizecodeframe}
    \frametitle{Write Concern}
    \item Write Concern allows users to specify at a Connection, DB, Collection or individual write level what the behavior should be with regards to \code{w}, \code{wtimeout}, and \code{fsync}
        \begin{itemize}
            \item \code{w}: \# of servers MongoDB must replicate the write to before returning 'OK'.  Default behavior is \code{0}, \code{1} waits for write and raises any errors as exceptions.  Setting it to \code{-1} ignores even network errors.
            \item \code{wtimeout}: \# of milliseconds to wait for the write to complete. Default behavior is \code{0}---wait forever.
            \item \code{fsync}: When \code{true}, forces MongoDB to sync the write to disk before returning.  Use with caution! Defaults to \code{false}.
        \end{itemize}
\end{itemizecodeframe}

\begin{itemizecodeframe}
    \frametitle{Write Concern}
    \framesubtitle{Constants}
    \item Casbah includes a Scala convenience wrapper in \code{com.mongodb.casbah.WriteConcern}, along with common values as predefs
        \begin{itemize}
            \item \code{WriteConcern.Normal}: Default behavior, raises exceptions for Network errors but not server errors.
            \item \code{WriteConcern.Safe}: Exceptions are raised for network issues and server errors, Casbah blocks and waits for the write to complete (equiv. to calling \code{getLastError})
            \item \code{Writeconcern.ReplicasSafe}: Exceptions are raised for network issues and server errors, Casbah blocks and waits for the write to complete on {\em at least} 2 servers in the replica set.
            \item \code{WriteConcern.FsyncSafe} Exceptions are raised for network issues and server errors, Casbah blocks and waits for the write to be flushed to disk.
        \end{itemize}
\end{itemizecodeframe}

\begin{itemizecodeframe}
    \frametitle{\code{request} mode}
    \item In the case where you want a single write to block for safety, there is \code{request} mode available on DB and Collection
    \item `Execute Around' pattern takes a function argument, calls \code{requestStart} and \code{requestDone} then invokes \code{getLastError.throwOnError} which throws an exception if errors occurred.  Your function is handed a copy of the DB or Collection at invocation.
    \item Example:\\
    \lstinputlisting{code/request.scala}
\end{itemizecodeframe}

\subsection{DSL Improvements}


\begin{itemizecodeframe}
    \frametitle{Query DSL Improvements}
    \framesubtitle{New Operators}
    \item Added support for all currently supported operators. A list of some of the new operators added in 2.0 include:
        \begin{itemize}
            \item             \$slice
            \item             \$or
            \item             \$not
            \item             \$each (special operator only supported nested inside `\$addToSet)
            \item             \$type (Uses type arguments and class manifests to allow a nice fluid Scala syntax)
            \item             \$elemMatch
            \item             Array Operators
            \item All GeoSpatial Operators including \$near and \$within
        \end{itemize}
\end{itemizecodeframe}

\begin{itemizecodeframe}
    \frametitle{Query DSL Improvements}
    \framesubtitle{Type Boundaries}
    \item Pushing harder on type safety for the DSL to prevent errors
    \item Type Safety is good but sometimes users introduce their own types
    \item Using Context Bounds w/ Type Classes to expand on this.
    \item Most operators now take a Type Class such as \code{ValidDateOrNumericType}:
        \lstinputlisting{code/neTypeClassDemo.scala}
    \pagebreak
    \item Type Classes are easily expanded, if you try to pass a non-working type.
        \lstinputlisting{code/newTypeClass.scala}
    
    \item Please let me know if I missed any boundaries on specific operators (e.g. I think I forgot to add Boolean support in a few places it should be, although \$ne isn't one of them)
\end{itemizecodeframe}
\end{document}
