\documentclass{beamer}

% Copyright 2010 by Brendan W. McAdams <bwmcadams@gmail.com>
%
% You may redistribute the content of this presentation for your own needs 
% provided you give credit to its author. In other words: "Don't be a dick."
% % Latex code based upon the "Generic Presentation 15-45 minutes" template from Beamer.


\mode<presentation>
{
  % \usetheme{Dresden}      
  \usetheme{CambridgeUS}      
  \usecolortheme{seahorse}
  \setbeamercovered{transparent}
}


\usepackage{hyperref}
\usepackage{xcolor}
\usepackage{listings}

\hypersetup{backref,  
pdfpagemode=FullScreen,  
colorlinks=false}

 % "define" Scala
\lstdefinelanguage{Scala}{
   morekeywords={abstract,case,catch,class,def,%
     do,else,extends,false,final,finally,%
     for,forSome,if,implicit,import,lazy,%
     match,mixin,%
     new,null,object,override,package,%
     private,protected,requires,return,sealed,%
     super,this,throw,trait,true,try,%
     type,val,var,while,with,yield},
   otherkeywords={=>,<-,<\%,<:,>:,\#,@},
   sensitive=true,
   morecomment=[l]{//},
   morecomment=[n]{/*}{*/},
   morestring=[b]",
   morestring=[b]',
   morestring=[b]"""
 }


 \lstdefinelanguage{JavaScript}{
    keywords={attributes, class, classend, do, empty, endif, endwhile, fail, function, functionend, if, implements, in, inherit, inout, not, of, operations, out, return, set, then, types, while, use},
    keywordstyle=\color{blue}\bfseries,
    ndkeywords={},
    ndkeywordstyle=\color{yellow}\bfseries,
    identifierstyle=\color{black},
    sensitive=false,
    comment=[l]{//},
    commentstyle=\color{green}\ttfamily,
    stringstyle=\color{red}\ttfamily
 }
\usepackage{caption}
\DeclareCaptionFont{white}{\color{white}}
\DeclareCaptionFormat{listing}{\colorbox{black}{\parbox{\textwidth}{#1#2#3}}}
\captionsetup[lstlisting]{format=listing,labelfont=white,textfont=white}


\usepackage{color}
\definecolor{dkgreen}{rgb}{0,0.6,0}
\definecolor{gray}{rgb}{0.5,0.5,0.5}
\definecolor{mauve}{rgb}{0.58,0,0.82}
 
% Default settings for code listings
\lstset{
  frame=tb,%
  language=Scala,%
  aboveskip=3mm,%
  belowskip=3mm,%
  xleftmargin=4mm,%
  showstringspaces=false,%
  keepspaces=true,%
  columns=[c]fixed,%
  basicstyle={\tiny\ttfamily},%
  escapechar=¤,%
  numbers=none,%
  numberstyle=\tiny\color{gray},%
  keywordstyle=\color{blue},%
  commentstyle=\color{dkgreen},%
  stringstyle=\color{mauve},%
  frame=single,%
  breaklines=true,%
  breakatwhitespace=true,%
  tabsize=2,%
  firstnumber=0,%
  numbersep=1.5mm,%
  numberstyle=\tiny%
}

% \lstdefinestyle{floating}{%
%     xleftmargin=10pt,%
%     xrightmargin=5pt,%
%     aboveskip=4mm,%
%     belowskip=4mm,%
%     fontadjust=true,%
%     columns=[c]flexible,%
%     keepspaces=true,%
%     basewidth={0.5em, 0.425em},%
%     tabsize=2,%
%     basicstyle=\renewcommand{\baselinestretch}{0.95}\ttfamily,%
%     commentstyle=\rm,%
%     keywordstyle=\bfseries,%
%     mathescape=true,%
%     captionpos=b,%
%     framerule=0.3pt,%
%     firstnumber=0,%
%     numbersep=1.5mm,%
%     numberstyle=\tiny,%
%     float=tbp,%
%     frame=tblr,%
%     framesep=5pt,%
%     framexleftmargin=3pt,%
%     abovecaptionskip=\smallskipamount,%
%     belowcaptionskip=\smallskipamount,%
% } % to define: caption, label

\newcommand{\code}[1]{%
    \lstinline[%keywordstyle=,%
               flexiblecolumns=true,%
               basicstyle=\ttfamily]£#1£}

\newenvironment{itemizeframe}
               {\begin{frame}\startitemizeframe} 
               {\stopitemizeframe\end{frame}}
              
\newenvironment{codeframe}
                {\begin{frame}[allowframebreaks,allowdisplaybreaks]}
                {\end{frame}}
                
\newenvironment{itemizecodeframe}
              {\begin{frame}[allowframebreaks,allowdisplaybreaks]
              \startitemizeframe} 
              {\stopitemizeframe\end{frame}}

\newcommand\startitemizeframe{\begin{itemize}} \newcommand\stopitemizeframe{\end{itemize}}

\usepackage[english]{babel}
% or whatever

\usepackage[latin1]{inputenc}
% or whatever

\usepackage{times}
\usepackage[T1]{fontenc}
% Or whatever. Note that the encoding and the font should match. If T1
% does not look nice, try deleting the line with the fontenc.


\title[Integrating Scala + MongoDB]{Scala with MongoDB} % (optional, use only with long paper titles)

%\subtitle
%{Presentation Subtitle} % (optional)

\author[B.W. McAdams]{Brendan W. McAdams}
% - Use the \inst{?} command only if the authors have different
%   affiliation.

\institute[Novus Partners]{Novus Partners, Inc.}
% - Use the \inst command only if there are several affiliations.
% - Keep it simple, no one is interested in your street address.

\date[NY Scala Enthusiasts - 8/8/10]
     {New York Scala Enthusiasts\\ Aug. 8, 2010}

\subject{Using Scala with MongoDB}
% This is only inserted into the PDF information catalog. Can be left
% out. 



% If you have a file called "university-logo-filename.xxx", where xxx
% is a graphic format that can be processed by latex or pdflatex,
% resp., then you can add a logo as follows:
\logo{\includegraphics[width=1.25cm]{novus-logo}}



% Delete this, if you do not want the table of contents to pop up at
% the beginning of each subsection:
\AtBeginSubsection[]
{
  \begin{frame}<beamer>
	\frametitle{Outline}
    \tableofcontents[currentsection,currentsubsection,currentsubsubsection]
  \end{frame}
}


% If you wish to uncover everything in a step-wise fashion, uncomment
% the following command: 

%\beamerdefaultoverlayspecification{<+->}


\begin{document}

\begin{frame}
  \titlepage
\end{frame}

\begin{frame}
\frametitle{Outline}
  \tableofcontents
  % You might wish to add the option [pausesections]
\end{frame}

\section{Introduction}
\subsection{Exposition}

\begin{itemizeframe}
\frametitle{Who Am I?}
    \item Started in Perl - lots of MySQL + Sybase (some PostgreSQL + Oracle)
	\item Spent time in C, PHP, Java, Python, C\#
	\item Built a bond trading platform with Perl and Java
	\item Last few years - lots of Python
	\item Last year - IronPython, C\#, MongoDB... 
	\item Research into NoSQL Tools led to MongoDB
\end{itemizeframe}


\begin{itemizeframe}
\frametitle{Briefly - Sluggy.com Rundown}
    \item I've maintained systems (and some code) for Sluggy.com since ~1999
    \item Original system was Perl + FreeBSD (flat files)
    \item Eventually migrated to Linux + PHP w/ MySQL
    \item Last Year: Rewrote with Python (Pylons) + MongoDB (pymongo + MongoKit), went live ~1 year ago.
    \item With Mongo, serving ~50GB/day (1.5TB of traffic/month on a single slicehost virtual machine)
    \item Opened my eyes to the new world of NoSQL
    \item See my talk from MongoNYC 2010 on our migration + what was learned
\end{itemizeframe}

\begin{itemizeframe}
\frametitle{Current Gig}
    \item Started at Novus Partners last fall
    \item Was toying with Scala while looking for work
    \item Dove into first project with Scala - haven't looked back.
    \item Existing Java development team now working in Scala
    \item Just expanded development team to another Scala developer
    \item Introduced MongoDB for rapid \& flexible data interaction for frontend.
    \item Several Open Source packages yielded...
        \begin{itemize}
            \item Casbah (mongoDB + Scala driver layer)
            \item Luau (mongoDB + Hadoop integration, written in Scala)
        \end{itemize}
\end{itemizeframe}

\subsection{What is MongoDB?}

\begin{itemizeframe}
\frametitle{Introducing MongoDB}
    \item Categorized as a ``Document-Oriented Database''
        \begin{itemize}
            \item<2-> Features of both Key-Value Stores \& RDBMS'
            \item<2-> Rich query interface.
            \item<2-> Works with JSON-like Documents
            \item<2-> Favors embedding related data over ``foreign key'' relationships
        \end{itemize}
    \item<3-> Free license (A-GPL) cross-platform (Packages for Linux, Windows, Mac OS X, Windows, FreeBSD \& Solaris)
    \item<4-> Cursor-based query results 
    \item<5-> Serverside Javascript
        \begin{itemize}
            \item<5-> Stored Javascript functions server-side
            \item<5-> Powerful aggregation - Map/Reduce, Group Commands
            \item<5-> JS Statements in queries (no indexes though)
        \end{itemize}
    \item<6-> Indexing system is much like RDBMS, includes Geospatial support.
    \item<7-> Scalable file storage with GridFS
    \item<8-> Data scalability with Replica Sets \& Autosharding
\end{itemizeframe}
\begin{itemizeframe}
\frametitle{Programming with MongoDB}
\item Provides a native API which allows interaction to adapt to the programming language (rather than vice versa).
\item<2-> Official drivers for...
    \begin{itemize}
        \item<2-> C
        \item<2-> C++
        \item<2-> Java
        \item<2-> JavaScript
        \item<2-> Perl
        \item<2-> PHP
        \item<2-> Python
        \item<2-> Ruby
    \end{itemize}
\item<3-> Community supported drivers include...
    \begin{itemize}
        \item<3-> .Net: C\# \& F\#
        \item<3-> JVM: Clojure, Scala, Groovy
        \item<3-> Erlang
        \item<3-> Haskell
        \item<3-> Go 
        \item<3-> ... and many more.
    \end{itemize}
\end{itemizeframe}
\begin{frame}
\frametitle{But is anyone actually *using* it?!?}
    MongoDB is deployed in production at companies including...
    \begin{itemize}
        \item New York Times 
        \item Foursquare 
        \item bit.ly 
        \item SourceForge
        \item Etsy 
        \item Disqus
        \item Github\pause
        \item ... The Large Hadron Collider.  
    \end{itemize}
\end{frame}

\subsection{A Taste of MongoDB}

\begin{frame}
\frametitle{The basics of Querying}
\lstinputlisting[language=javascript]{code/querying.js}
\end{frame}

\begin{frame}
\frametitle{Query Objects}
\lstinputlisting[language=javascript]{code/queryobjects.js}
\end{frame}

\begin{frame}
\frametitle{Geospatial Support}
\lstinputlisting[language=javascript]{code/geospatial.js}
\end{frame}

\begin{itemizeframe}
\frametitle{Finally, Data Scalability.}
\item Replica Sets
\item AutoSharding
\end{itemizeframe}

\subsection{MongoDB + Scala Drivers}

\begin{itemizecodeframe}
\frametitle{Using Scala with the official Java Driver}
\item JVM Object are JVM Objects...
\lstinputlisting{code/javadriver_sample.scala}
\item ... Not terribly ``Scala-ey''.
\item The Java driver works, but doesn't fit well in Scala.
\item You need to convert your Scala objects to Java Objects, and get nothing but Java Objects out.
\item Gets messy quickly.
\end{itemizecodeframe}

\begin{codeframe}
\frametitle{The Scala Community Adapted...}
Compare the previous with various Scala drivers.
\begin{itemize}
    \item mongo-scala-driver wraps \& enhances the Java driver:
        \lstinputlisting{code/mongo-scala-driver_sample.scala}
    \item .. Much better, although I was confused initially.  Has a object<->MongoDB mapping layer.
    \item lift-mongodb has more than one way to do it... here's just a taste:
        \lstinputlisting{code/lift-mongodb_sample.scala}
    \item ... Lift's JS \& JSON tools make it very flexible, as we'll see later.  Also has an ActiveRecord style Object<->MongoDB Mapping layer.
    \item Casbah reflects my own attempt at creating a sane interface between Scala \& MongoDB.  Influenced by pymongo:
        \lstinputlisting{code/casbah_sample.scala}
    \item ... The syntax is still growing but is meant to match Scala syntax sanely.  Object<->MongoDB Mapping coming soon.
    \item We're going to cover several tools, although I know Casbah best.
\end{itemize}
\end{codeframe}

\section{Interlude: Helping Scala + Java play nice together.}

\begin{itemizeframe}
    \frametitle{Helping Java + Scala Interact}
    \item Implicits, ``Pimp My Library'' and various conversion helper tools simplify the work of interacting with Java.
    \item<2-> Scala and Java have their own completely different collection libraries. 
    \item<3-> Some builtins ship with Scala to make this easier.
\end{itemizeframe}
\subsection{Java <-> Scala Basics}

\begin{itemizeframe}
    \frametitle{Interoperability in Scala 2.7.x}
    \item Scala 2.7.x shipped with \texttt{scala.collection.jcl}.
    \item<2-> \texttt{scala.collection.jcl.Conversions} contained some implicit converters, but only to and from the wrapper versions - no support for ``real'' Scala collections.
    \item<3-> Neglected useful base interfaces like \texttt{Iterator} and \texttt{Iterable}
    \item<3-> @jorgeortiz85 provided \texttt{scala-javautils}, which used ``Pimp My Library'' to do a better job.
\end{itemizeframe}

\begin{itemizeframe}
    \frametitle{Interoperability in Scala 2.8.x}
    \item Scala 2.8.x improves the interop game significantly.
    \item<2-> JCL is gone - focus has shifted to proper interoperability w/ built-in types.
    \item<3-> \texttt{scala.collection.jcl.Conversions} replaced by \texttt{scala.collection.JavaConversions} - provides implicit conversions to \& from Scala \& Java Collections.
    \item<3-> Includes support for the things missing in 2.7 (\texttt{Iterable}, \texttt{Iterator}, etc.)
    \item<3-> Great for places where the compiler can guess what you want (implicits); falls short in some cases (like BSON Encoding, as we found in Casbah)
    \item<4-> @jorgeortiz85 has updated \texttt{scala-javautils} for 2.8 with  \texttt{scalaj-collection}
    \item<4-> Explicit \texttt{asJava} / \texttt{asScala} methods for conversions.  Adds \texttt{foreach} method to Java collections.
\end{itemizecodeframe}

\subsection{Implicits and Pimp Hats}
\begin{itemizeframe}
    \frametitle{So WTF is an `Implicit', anyway?}
    \item Implicit Arguments
        \begin{itemize}
        \item `Explicit' arguments indicates a method argument you pass, well \emph{explicitly}.
        \item `Implicit' indicates a method argument which is... \emph{implied}. (But you can pass them explicitly too.)
        \item Implicit arguments are passed in Scala as an additional argument list:
            \lstinputlisting{code/implicit_sample_arg1.scala}
        \pause
        \item How does this differ from default arguments?
        \end{itemize}
\end{itemizeframe}

\begin{itemizeframe}
    \frametitle{So WTF is an `Implicit', anyway?}
    \item Implicit Methods/Conversions
        \begin{itemize}
            \item If you try passing a type to a Scala method argument which  doesn't match...
                \lstinputlisting{code/implicit_sample_method1.scala}
                
            \item<2-> A fast and loose example, but simple.  Fails to compile.
            \item<3-> But with implicit methods, we can provide a conversion path...
                \lstinputlisting{code/implicit_sample_method2.scala}
            \item<4-> In a dynamic language, this may be called ``monkey patching''. Unlike Perl, Python, etc. Scala resolves implicits at compile time.
        \end{itemize}
\end{itemizeframe}

\begin{itemizeframe}
    \frametitle{Pimp My Library}
    \item Coined by Martin Odersky in a 2006 Blog post.  Similar to C\# extension methods, Ruby modules.
    \item<2-> Uses implicit conversions to tack on new methods at runtime.
    \item<3-> Either return a new ``Rich\_'' or anonymous class...
        \lstinputlisting{code/pimp_sample1.scala}
    \item<3-> A note: with regards to type boundaries, \texttt{[A <: SomeType]} won't allow implicitly converted values.  You can whitelist them by using \texttt{[A <\% SomeType]} instead.
\end{itemizeframe}
\section{Scala + MongoDB == Win}
\subsection{mongo-scala-driver}
\subsubsection{Shapes for Object Mapping}

\subsection{lift-mongo}
\subsubsection{lift-mongo-record for Object Mapping}
\subsubsection{Foursquare's approach to better querying}

\subsection{casbah}
\begin{itemizeframe}
    \frametitle{Shameless Self Promotion}
    \item Why Casbah?
	\item     Background in pymongo + MongoKit 

	\item     Java driver too... ``Java-ey''

	\item     Didn't quite ``get'' scamongo and mongo-scala-driver early on

	\item     scamongo's base didn't fix most of my issues w/ the Java Driver (just helped connection management)

	\item     scamongo's ORM libraries were dependent on Lift (now scamongo is defunct and has become lift-mongo)

	\item     mongo-scala-driver's shapes, etc were \emph{very} confusing to me as a newbie w/o much functional background

\end{itemizeframe}

\begin{itemizeframe}
    \frametitle{Casbah is Born}
    \item Borrowed bits I liked/understood from other places and built something that felt comfortable to me 
    \item Early on, very pythonic
    \item Query DSL, grown from wanting a feel close to the ``metal'' based on generic MongoDB knowledge
    \item Heavily influenced in structure by \texttt{jorgeortiz85}'s libraries
    \item Quickly grew as I used more and more MongoDB with Scala; features have been grown organically from my own needs.
\end{itemizeframe}

\begin{itemizecodeframe}
    \frametitle{Interacting with DBObjects}
    \item \texttt{DBObject} is far too structurally Java.
    \item Sought to make them more usable \& readable from Scala
    \item Most recently - match Scala 2.8 collection Factory/Builders
    \item Implicit conversions of \texttt{Product} (base for Tuple), \texttt{Map}.  Explicit method \texttt{asDBObject} for corner cases.
    \item `Pimped' version of \texttt{DBObject} via \texttt{MongoDBObject} - lets DBObject implement Scala's \texttt{Map} trait.
        \lstinputlisting{code/casbah_dbobject_sample1.scala}
    \item DBCollection behaves as a Scala \texttt{Iterable}, but interaction is mostly the same (with addition of methods like \texttt{+=}).
\end{itemizecodeframe}

\begin{itemizecodeframe}
    \frametitle{Fluid Query Syntax}
    \item My thought: Instead of keeping track of \textbf{Yet Another API}, MongoDB's Query Objects should ``just work''.
    \item Two kinds of Query Operators - `Bareword' and `Core'.
    \item Bareword Operators can be started as `bare' statements:
        \lstinputlisting{code/casbah_bareword_ops1.scala}
    \item Core Operators need to be anchored to the right of a \texttt{DBObject} or a String (typically representing a field name):
        \lstinputlisting{code/casbah_core_ops1.scala}
    \item Just a small taste - all MongoDB Query Objects are supported (For 1.4.x syntax - 1.6.x (\$or, etc. soon))
\end{itemizecodeframe}

\begin{itemizecodeframe}
    \frametitle{Other Features}
    \item Custom convertor implementations which allow most Scala types to be serialized cleanly to MongoDB.  (Joda time serialization/deserialization support).
    \item Improved GridFS Functionality (loan pattern, support for \texttt{scala.io.Source})
    \item Wrapper objects for Map/Reduce system (Help parse results to warn of errors, etc)
\end{itemizecodeframe}

\begin{itemizecodeframe}
    \frametitle{Coming Soon}
    \item Max Afonov {\scriptsize @max4f} working on annotation driven object mapping.
    \item Investigating ActiveRecord implementation, with fluid query syntax support.
    \item Support for MongoDB 1.6.x features.
\end{itemizecodeframe}

\subsection{STM + MongoDB via Akka}
\begin{itemizecodeframe}
    \frametitle{STM + MongoDB via Akka}
    \item Akka has an implementation of STM inspired by Clojure's; allows datastructures such as Maps and Vectors to become transactional.
    \item Akka STM supports persistence to several backends including MongoDB.
    \item Allows you to setup relatively simple, code managed concurrent transactions with state stored safely in MongoDB.
    \item Supports JTA; not yet distributed (Dependent on Multiverse, which is working on distributed STM)
\end{itemizecodeframe}

\section{Closing}
\begin{itemizeframe}
    \frametitle{Links}
    \item mongo-scala-driver {\scriptsize http://github.com/alaz/mongo-scala-driver}
    \item lift-mongo {\scriptsize http://www.assembla.com/wiki/show/liftweb/MongoDB}
    \item FourSquare's Lift Mongo DSL Code ... coming soon? {\scriptsize @jliszka}
    \item Casbah {\scriptsize http://novus.github.com/docs/casbah}
    \item Jorge Ortiz' ({\scriptsize @jorgeortiz85}) Libraries
        \begin{itemize}
            \item scala-javautils (Scala 2.7.x) {\scriptsize http://github.com/jorgeortiz85/scala-javautils}
            \item scalaj-collection (Scala 2.8.x) {\scriptsize http://github.com/scalaj/scalaj-collection}
        \end{itemize}
    \item This presentation {\scriptsize http://github.com/bwmcadams/presentations/tree/master/scala\_mongodb/scalany\_aug10/}
\end{itemizeframe}


\begin{itemizeframe}
    \frametitle{Contact Info}
    \item Twitter: {\scriptsize @rit}
    \item Email: {\scriptsize bwmcadams@gmail.com}
    \item Github: {\scriptsize http://github.com/bwmcadams} | {\scriptsize http://github.com/novus}
    \item IRC - freenode.net \#mongodb
    \item MongoDB Mailing List {\scriptsize http://groups.google.com/group/mongodb-user}
    \item Casbah Mailing List {\scriptsize http://groups.google.com/group/mongodb-casbah-user}
    \item Boston MongoDB Conference - Sept. 20 (Cambridge, Mass.) {\scriptsize http://10gen.com/conferences/mongoboston2010}
    \item MongoDB NY Users Group {\scriptsize http://www.meetup.com/New-York-MongoDB-User-Group/}
\end{itemizeframe}

\end{document}


