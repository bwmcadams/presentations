\documentclass[xcolor=dvipsnames]{beamer}

% Copyright 2010 by Brendan W. McAdams <bwmcadams@gmail.com>
%
% You may redistribute the content of this presentation for your own needs 
% provided you give credit to its author. In other words: "Don't be a dick."
% % Latex code based upon the "Generic Presentation 15-45 minutes" template from Beamer.


\mode<presentation>
{
  % \usetheme{Dresden}      
  \usetheme{Madrid}
  \setbeamercolor{structure}{fg=OliveGreen!125!black} 
        
  % \usecolortheme{dolphin}
  \setbeamercovered{transparent}
}


\usepackage{hyperref}
\usepackage{xcolor}
\usepackage{listings}

\hypersetup{backref,  
pdfpagemode=FullScreen,  
colorlinks=false}

 % "define" Scala
\lstdefinelanguage{Scala}{
   morekeywords={abstract,case,catch,class,def,%
     do,else,extends,false,final,finally,%
     for,forSome,if,implicit,import,lazy,%
     match,mixin,%
     new,null,object,override,package,%
     private,protected,requires,return,sealed,%
     super,this,throw,trait,true,try,%
     type,val,var,while,with,yield},
   otherkeywords={=>,<-,<\%,<:,>:,\#,@},
   sensitive=true,
   morecomment=[l]{//},
   morecomment=[n]{/*}{*/},
   morestring=[b]",
   morestring=[b]',
   morestring=[b]"""
 }


\lstdefinelanguage{JavaScript}{
    keywords={typeof, new, true, false, catch, function, return, null, catch, switch, var, if, in, while, do, else, case, break}
    keywordstyle=\color{blue}\bfseries,
    ndkeywords={class, export, boolean, throw, implements, import, this},
    ndkeywordstyle=\color{darkgray}\bfseries,
    identifierstyle=\color{black},
    sensitive=false,
    comment=[l]{//},
    morecomment=[s]{/*}{*/},
    commentstyle=\color{purple}\ttfamily,
    stringstyle=\color{red}\ttfamily,
    morestring=[b]',
    morestring=[b]"
}
\usepackage{caption}
\DeclareCaptionFont{white}{\color{white}}
\DeclareCaptionFormat{listing}{\colorbox{black}{\parbox{\textwidth}{#1#2#3}}}
\captionsetup[lstlisting]{format=listing,labelfont=white,textfont=white}


\usepackage{color}
\definecolor{dkgreen}{rgb}{0,0.6,0}
\definecolor{gray}{rgb}{0.5,0.5,0.5}
\definecolor{mauve}{rgb}{0.58,0,0.82}
\definecolor{lightgray}{rgb}{.9,.9,.9}
\definecolor{darkgray}{rgb}{.4,.4,.4}
\definecolor{purple}{rgb}{0.65, 0.12, 0.82}

% Default settings for code listings
\lstset{
  frame=tb,%
  language=Scala,%
  aboveskip=1.5mm,%
  belowskip=1.5mm,%
  xleftmargin=1mm,%
  xrightmargin=1mm,%
  showstringspaces=false,%
  keepspaces=true,%
  columns=[c]fixed,%
  basicstyle={\tiny\ttfamily},%
  escapechar=¤,%
  numbers=none,%
  numberstyle=\tiny\color{gray},%
  keywordstyle=\color{blue},%
  commentstyle=\color{dkgreen},%
  stringstyle=\color{mauve},%
  frame=single,%
  breaklines=true,%
  breakatwhitespace=true,%
  tabsize=2,%
  firstnumber=0,%
  numbersep=1.5mm,%
  numberstyle=\tiny%
}

\newcommand{\code}[1]{%
    \lstinline[%keywordstyle=,%
               flexiblecolumns=true,%
               basicstyle=\ttfamily]£#1£}

\newenvironment{itemizeframe}
               {\begin{frame}\startitemizeframe} 
               {\stopitemizeframe\end{frame}}
              
\newenvironment{codeframe}
                {\begin{frame}[allowframebreaks,allowdisplaybreaks]}
                {\end{frame}}
                
\newenvironment{itemizecodeframe}
              {\begin{frame}[allowframebreaks,allowdisplaybreaks]
              \startitemizeframe} 
              {\stopitemizeframe\end{frame}}

\newcommand\startitemizeframe{\begin{itemize}} \newcommand\stopitemizeframe{\end{itemize}}

\usepackage[english]{babel}
% or whatever

\usepackage[latin1]{inputenc}
% or whatever

\usepackage{times}
\usepackage[T1]{fontenc}
% Or whatever. Note that the encoding and the font should match. If T1
% does not look nice, try deleting the line with the fontenc.


\title[Casbah: Scala + MongoDB Integration]{Casbah: Scala + MongoDB Integration} % (optional, use only with long paper titles)

%\subtitle
%{Presentation Subtitle} % (optional)

\author[Brendan McAdams]{Brendan W. McAdams \\
        {\tiny brendan@10gen.com} \\
        {\scriptsize @rit } }
% - Use the \inst{?} command only if the authors have different
%   affiliation.

\institute[10gen]{10gen, Inc.}
% - Use the \inst command only if there are several affiliations.
% - Keep it simple, no one is interested in your street address.

\date[Chicago Scala - 10/21/10]
     {Chicago Scala Meetup - Oct. 21, 2010}

\subject{Using Scala with MongoDB}

\logo{\includegraphics[width=2.25cm]{mongodb-logo}}


% Delete this, if you do not want the table of contents to pop up at
% % the beginning of each subsection:
% \AtBeginSubsection[]
% {
%   \begin{frame}<beamer>
%   \frametitle{Outline}
%     \tableofcontents[currentsection,currentsubsection,currentsubsubsection]
%   \end{frame}
% }


% If you wish to uncover everything in a step-wise fashion, uncomment
% the following command: 

%\beamerdefaultoverlayspecification{<+->}


\begin{document}

\begin{frame}
  \titlepage
\end{frame}

\begin{frame}
\frametitle{Outline}
  \tableofcontents
  % You might wish to add the option [pausesections]
\end{frame}

\section{Introduction to MongoDB}

\subsection{What is MongoDB?}

\begin{itemizecodeframe}
    \frametitle{What Is MongoDB?}
    
    \item ``Document-Oriented'' Database, with feature of both Key-Value Stores \& RDBMS'
        \begin{itemize}
            \item Rich Query Interface
            \item Works with JSON-like documents
            \item Favors embedding data over ``foreign key'' relationships
        \end{itemize}
    \pagebreak
    \item Open Source \& Free: Server is licensed under A-GPL, Official language drivers under Apache 2
    \item Focused on native APIs for MongoDB interaction to adapt to the host language's native idioms (rather than vice versa) 
        \begin{itemize}
            \item Official Drivers for\dlots C, C\#, C++, Java, JavaScript, Perl, PHP, Python, Ruby 
            \item Community Supported drivers for\ldots Clojure, Scala, Erlang, F\#, Go, Haskell, Lua, Objective C, Smalltalk and more\ldots
        \end{itemize}
    \pagebreak
    \item Cursor-based query results
    \item ServerSide JavaScript 
        \begin{itemize}
            \item Stored JavaScript functions server-side
            \item Powerful aggregation via Map/Reduce \& Group Commands
            \item JavaScript statements in queries (No indexes, though)
        \end{itemize}
    \item Indexing system much like RDBMS', includes Geospatial support
    \item Scalable file storage with GridFS
    \item Data scalability with Replica Sets \& Sharding
\end{itemizecodeframe}


\begin{itemizeframe}
    \frametitle{But is anyone actually *using* it?!?}
    \item MongoDB is deployed in production at companies which include\ldots
    \begin{itemize}
        \item Foursquare
        \item Sourceforge
        \item Etsy
        \item The New York Times
        \item Justin.tv
        \item Github
        \item Chicago Tribune (See: \url{http://schools.chicagotribune.com/})
        \item Scrabb.ly (Scrabble MMO built in 48 hours using MongoDB geospatial indexes to determine tile placement)
        \item Many, many more (including the Large Hadron Collider)
    \end{itemize}
    \item A list which includes details on various deployments is available at \url{http://www.mongodb.org/display/DOCS/Production+Deployments}
\end{itemizeframe}

\subsection{A Taste of MongoDB}
\begin{itemizecodeframe}
    \frametitle{Core Concepts}
    \item MongoDB's equivalent to ``tables'' are referred to as ``collections'', which contain ``documents'' (individual pieces of data)
    \item DBs \& Collections are lazy - they are created when first written to
    \item MongoDB's wire format/internal representation is {\bf BSON} - Binary JSON
    \begin{itemize}
        \item Binary optimized flavor of {\bf JSON}; corrects several shortfalls.
        \item Binary efficient string encoding ({\bf JSON uses Base64})
        \item Supports other features such as Regular Expressions, Byte Arrays, DateTimes \& Timestamps and JavaScript code blocks \& functions.
        \item Implemented in separate packages for official drivers
        \item Createive Commons Licensed, available at \url{http://bsonspec.org}
    \end{itemize}
    \item Java driver (which Casbah wraps) represents {\bf BSON} objects with a map-like {\bf DBObject}; many dynamic languages (Perl, Python, etc.) use native dictionary objects.
\end{itemizecodeframe}




\begin{itemizecodeframe}
\frametitle{The basics of Querying}
    \item Find a single row with \emph{findOne()}; returns the first document found (by natural order).  
    \item You can find all documents matching your query with \emph{find()}.  No query means you get the entire collection back.
    \item Queries are specified as {\bf BSON} documents to match against.
    \pagebreak
    \item The \emph{find()} and \emph{findOne()} methods can take an optional second {\bf DBObject} specifying the fields to return.
    \item If you have an embedded object (for example, an address object) you can retrieve it with dot notation in the fields list (e.g. {\em ``address.city''} retrieves just the city value).

    \item Use {\em limit()}, {\em skip()} and {\em sort()} on result objects ({\bf DBCursor} in Java-driver land) to adjust your results.  These all return a new cursor.
    \pagebreak
    \item {\em distinct()} can be used (on {\bf DBCollection} to find all distinct values for a given key; it returns a list of values.
\lstinputlisting[language=JavaScript]{code/querying.js}
\end{itemizecodeframe}

\begin{itemizecodeframe}
\frametitle{Query Operators}
 \item MongoDB is no mere Key-Value store. There are myriad powerful operators to enhance your MongoDB queries\ldots
\begin{itemize}{\small
    \item Conditional Operators: {\bf \$gt} (>), {\bf \$lt} (<), {\bf \$gte (>=)},  {\bf \$lte (<=)}
    \item Negative Equality: {\bf \$ne} (!=) 
    \item Array Operators: {\bf \$in} (SQL ``IN'' clause\ldots takes an array), {\bf \$nin} (Opposite of ``IN''), {\bf \$all} (Requires all values in the array match), {\bf \$size} (Match the size of an array)
    \item Field Defined: {\bf \$exists} (boolean argument)(Great in a schemaless world)
    \item Regular Expressions (Language dependent - most drivers support it)
    \item Pass Arbitrary Javascript with {\bf \$where}, boolean OR with {\bf \$or}
    \item Negate any operator with {\bf \$not}
    }
\end{itemize}
\item Using a query operator requires nested objects\ldots
\lstinputlisting[language=JavaScript]{code/queryoperators_sample.js}
\item No syntactic sugar in Java to make it easier\ldots
\end{itemizecodeframe}

\begin{itemizecodeframe}
    \frametitle{Insert/Update/Save}
    \item Objects in MongoDB Collections have an ``\_id'' field, which must be unique.
    \item Three ways to add/update data in MongoDB\ldots
        \begin{itemize}
            \item \texttt{insert()} always attempts to add a new row.  If ``\_id'' is present and contains a value already in the collection, insert fails.
            \item \texttt{save()} inserts if there is no ``\_id'' field, otherwise it tries to update the document with the specified ``\_id''.
            \item \texttt{update()} takes a query and the new values to save.  By default it updates only the first document matching the query.
            \item For \texttt{update()} you can specify two booleans whose default is false: \emph{upsert}, which indicates you wish to create a new document if the query doesn't match,  and \emph{multi}, which allows updating \textbf{all} documents who match the query.
        
        \end{itemize}
    \lstinputlisting[language=JavaScript]{code/mongo_insert.js}
\end{itemizecodeframe}


\begin{itemizecodeframe}
\frametitle{Geospatial Support}
\item MongoDB supports Geospatial indexing and distance based queries
\item I loaded all of the NYC Subway data (in Google Transit format) into MongoDB
\item Quick python code to index the ``Stops'' data.
\lstinputlisting[language=Python]{code/geoindex_snippet.py}
\item ``stop\_geo'' field is now Geospatially indexed for each stop.
\item How hard is it to find the 5 closest subway stops to Meetup HQ?
\end{itemizecodeframe}
\begin{itemizecodeframe}
\frametitle{Geospatial Support}
\lstinputlisting[language=JavaScript]{code/geospatial.js}
\item Further commands exist to define a rectangle or circle radius for the search.  
\end{itemizecodeframe}

\begin{itemizeframe}
\frametitle{Finally, Data Scalability.}
\item Traditional master-slave replication
\item Replica Sets (new in 1.6)
    \begin{itemize}
        \item Replaces master-slave setup with 1-7 server clusters
        \item Automatic failover and recovery
    \end{itemize}
\item AutoSharding (new in 1.6)
    \begin{itemize}
        \item Horizontal scaling - partition your collections \& data across as many nodes as necessary.
        \item Multiple nodes can service the same shard, allowing for balancing \& failover.
        \item Map/Reduce runs across multiple shards, allowing concurrency.
    \end{itemize}
\end{itemizeframe}
\section{A few Scala tidbits\ldots}

\begin{itemizecodeframe}
    \frametitle{User Defined Operators}
    \item Scala allows the definition of operators...
    \item Not technically operator overloading, as the JVM doesn't have operators - they're language built-ins in Java, etc.  
    \item In Scala, there are no built-in operators.  Some are predefined for sanity (like \texttt{+}, \texttt{-}, \texttt{/} and \texttt{*} on Numeric types)  but operators are just methods.
    \item Scala allows any operator to be defined by the user - including some special ones like unaries (\texttt{+foo}, \texttt{-foo}).
    \item Syntactic Sugar: To facilitate statements like \texttt{foo + bar} Scala allows methods to be called without the \texttt{.} or parentheses\ldots Useful for DSLs and fluid syntaxes!
    \lstinputlisting{code/operators.scala}
\end{itemizecodeframe}

\subsection{Functional Programming Briefly}

\begin{itemizecodeframe}
    \frametitle{Functional Programming and Scala}
    \item What is Functional Programming?
        \begin{itemize}
            \item Functions are Objects
            \item Immutability
        \end{itemize}
    \item A few crucial Scala concepts which depend upon FP (and Scala programmers delight in)
        \begin{itemize}
            \item Anonymous Functions
            \item \texttt{apply()} (and \texttt{unapply})
            \item Useful built-in Collection methods like \texttt{group}, \texttt{foreach} and \texttt{map}
        \end{itemize}
\end{itemizecodeframe}

\subsection{Java <-> Scala Basics}
\begin{itemizeframe}
    \frametitle{Helping Java + Scala Interact}
    \item Implicits, ``Pimp My Library'' and various conversion helper tools simplify the work of interacting with Java.
    \item Scala and Java have their own completely different collection libraries. 
    \item Some builtins ship with Scala to make this easier.
\end{itemizeframe}


\begin{itemizeframe}
    \frametitle{Interoperability in Scala 2.7.x}
    \item Scala 2.7.x shipped with \texttt{scala.collection.jcl}.
    \item \texttt{scala.collection.jcl.Conversions} contained some implicit converters, but only to and from the wrapper versions - no support for ``real'' Scala collections.
    \item Neglected useful base interfaces like \texttt{Iterator} and \texttt{Iterable}
    \item @jorgeortiz85 provided \texttt{scala-javautils}, which used ``Pimp My Library'' to do a better job.
\end{itemizeframe}

\begin{itemizeframe}
    \frametitle{Interoperability in Scala 2.8.x}
    \item Scala 2.8.x improves the interop game significantly\ldots JCL is gone in favor of conversions for builtin types
    \item \texttt{scala.collection.jcl.Conversions} replaced by \texttt{scala.collection.JavaConversions} - provides implicit conversions to \& from Scala \& Java Collections.
    \item Includes support for the things missing in 2.7 (\texttt{Iterable}, \texttt{Iterator}, etc.)
    \item Great when the compiler can guess what you want (implicits); falls short in all other cases (like BSON Encoding, as we found in Casbah)
    \item @jorgeortiz85 has updated \texttt{scala-javautils} for 2.8 with  \texttt{scalaj-collection}
    \item Explicit \texttt{asJava} / \texttt{asScala} methods for conversions.  Adds \texttt{foreach} method to Java collections.
\end{itemizecodeframe}

\subsection{Implicits and Pimp Hats}
\begin{itemizeframe}
    \frametitle{So WTF is an `Implicit', anyway?}
    \item Implicit Arguments
        \begin{itemize}
        \item `Explicit' arguments indicates a method argument you pass, well \emph{explicitly}.
        \item `Implicit' indicates a method argument which is\ldots \emph{implied}. (But you can pass them explicitly too.)
        \item Implicit arguments are passed in Scala as an additional argument list:
            \lstinputlisting{code/implicit_sample_arg1.scala}
        \pause
        \item How does this differ from default arguments?
        \end{itemize}
\end{itemizeframe}

\begin{itemizeframe}
    \frametitle{So WTF is an `Implicit', anyway?}
    \item Implicit Methods/Conversions
        \begin{itemize}
            \item If you try passing a type to a Scala method argument which  doesn't match\ldots
                \lstinputlisting{code/implicit_sample_method1.scala}
                
            \item A fast and loose example, but simple.  Fails to compile.
            \item But with implicit methods, we can provide a conversion path\ldots
                \lstinputlisting{code/implicit_sample_method2.scala}
            \item In a dynamic language, this may be called ``monkey patching''. Unlike Perl, Python, etc. Scala resolves implicits at compile time.
        \end{itemize}
\end{itemizeframe}

\begin{itemizeframe}
    \frametitle{Pimp My Library}
    \item Coined by Martin Odersky in a 2006 Blog post.  Similar to C\# extension methods, Ruby modules.
    \item Uses implicit conversions to tack on new methods at runtime.
    \item Either return a new ``Rich\_'' or anonymous class\ldots
        \lstinputlisting{code/pimp_sample1.scala}
\end{itemizeframe}

\subsection{MongoDB + Scala}

\begin{itemizecodeframe}
\frametitle{Using Scala with the official Java Driver}
\item DBObjects are JVM Objects\ldots
\lstinputlisting{code/javadriver_sample.scala}
\item \ldots Not terribly ``Scala-ey''.
\item The Java driver works, but doesn't fit well in Scala.
\item You need to convert your Scala objects to Java Objects, and get nothing but Java Objects out.
\item Gets messy quickly.
\end{itemizecodeframe}

\begin{codeframe}
\frametitle{The Scala Community Adapted\ldots}
Compare the previous with various Scala drivers.
\begin{itemize}
    \item mongo-scala-driver wraps \& enhances the Java driver:
        \lstinputlisting{code/mongo-scala-driver_sample.scala}
    \item .. Much better, although I was confused initially.  Has a object<->MongoDB mapping layer.
    \pagebreak
    \item lift-mongodb has more than one way to do it\ldots here's just a taste:
        \lstinputlisting{code/lift-mongodb_sample.scala}
    \item \ldots Lift's JS \& JSON tools make it very flexible.  Also has an ActiveRecord style Object<->MongoDB Mapping layer.
    \pagebreak
    \item Casbah reflects my own attempt at creating a sane interface between Scala \& MongoDB.  Influenced by pymongo:
        \lstinputlisting{code/casbah_sample.scala}
    \item \ldots The syntax is meant to match Scala syntax \& idioms sanely.

\end{itemize}
\end{codeframe}

\section{Scala + MongoDB == Win}


\subsection{lift-mongo}
\begin{itemizecodeframe}
    \frametitle{lift-mongo}
    \item Formerly ``scamongo'', integrated with Lift as of 2.0 
    \item Base code provides session wrappers to MongoDB, still utilizes Java driver's DBObject code.
            \lstinputlisting{code/lift-mongodb-base_sample.scala}
    \item ``lift-mongo-record'' provides object mapping.
    \item No native query syntax, but Foursquare is working on open sourcing something they use internally.
\end{itemizecodeframe}

\begin{itemizecodeframe}
    \frametitle{lift-mongo-record \& querying}
    \item Object definitions are fairly straightforward\ldots
            \lstinputlisting{code/lift-mongodb-record_sample.scala}
    \pagebreak
    \item Foursquare's query library allow for a saner way to query data\ldots 
            \lstinputlisting{code/lift-mongodb-4sq_sample.scala}
    \item \emph{Thank you} to \texttt{@jliszka} for sharing this!
\end{itemizecodeframe} 
 
\subsection{casbah}
\begin{itemizeframe}
    \frametitle{Shameless Self Promotion}
    \item Why Casbah?
	\item     Background in pymongo + MongoKit 

	\item     Java driver too\ldots ``Java-ey''

	\item     Didn't quite ``get'' scamongo and mongo-scala-driver early on

	\item     scamongo's base didn't fix most of my issues w/ the Java Driver (just helped connection management)

	\item     scamongo's ORM libraries were dependent on Lift (now scamongo is defunct and has become lift-mongo)

	\item     mongo-scala-driver's shapes, etc were \emph{very} confusing to me as a newbie w/o much functional background

\end{itemizeframe}

\begin{itemizeframe}
    \frametitle{Casbah is Born}
    \item Borrowed bits I liked/understood from other places and built something that felt comfortable to me 
    \item Early on, very pythonic
    \item Query DSL, grown from wanting a feel close to the ``metal'' based on generic MongoDB knowledge
    \item Heavily influenced in structure by \texttt{@jorgeortiz85}'s libraries
    \item Quickly grew as I used more and more MongoDB with Scala; features have been grown organically from my own needs.
\end{itemizeframe}

\begin{itemizecodeframe}
    \frametitle{Interacting with DBObjects}
    \item \texttt{DBObject} is far too structurally Java.
    \item Sought to make them more usable \& readable from Scala
    \item Most recently - match Scala 2.8 collection Factory/Builders
    \item Implicit conversions of \texttt{Product} (base for Tuple), \texttt{Map}.  Explicit method \texttt{asDBObject} for corner cases.
    \pagebreak
    \item `Pimped' version of \texttt{DBObject} via \texttt{MongoDBObject} - lets DBObject implement Scala's \texttt{Map} trait.
        \lstinputlisting{code/casbah_dbobject_sample1.scala}
    \item DBCollection behaves as a Scala \texttt{Iterable}, but interaction is mostly the same (with addition of methods like \texttt{+=}).
        \lstinputlisting{code/casbah_iteration_example.scala}
\end{itemizecodeframe}

\begin{itemizecodeframe}
    \frametitle{Fluid Query Syntax}
    \item My thought: Instead of keeping track of \textbf{Yet Another API}, MongoDB's Query Objects should ``just work''.
    \item Two kinds of Query Operators - `Bareword' and `Core'.
    \item Bareword Operators can be started as `bare' statements:
        \lstinputlisting{code/casbah_bareword_ops1.scala}\pagebreak
    \item Core Operators need to be anchored to the right of a \texttt{DBObject} or a String (typically representing a field name):
        \lstinputlisting{code/casbah_core_ops1.scala}
    \item Just a small taste - all MongoDB Query Objects are supported (For 1.4.x syntax - 1.6.x (\$or, etc. soon))
\end{itemizecodeframe}

\begin{itemizecodeframe}
    \frametitle{Other Features}
    \item Custom converter implementations which allow most Scala types to be serialized cleanly to MongoDB.  (Joda time serialization/deserialization support).
    \item Improved GridFS Functionality (loan pattern, support for \texttt{scala.io.Source})
        \lstinputlisting{code/gridfs_loan.scala}
    \item Wrapper objects for Map/Reduce system (Help parse results to warn of errors, etc)
        \lstinputlisting{code/map_reduce.scala}
    
\end{itemizecodeframe}

\begin{itemizecodeframe}
    \frametitle{Coming Soon}
    \item Max Afonov {\scriptsize @max4f} working on annotation driven object mapping ({\bf casbah-mapper})
    \item Investigating ActiveRecord implementation, with fluid query syntax support.
    \item Support for MongoDB 1.6.x features.
\end{itemizecodeframe}

\begin{codeframe}
    \frametitle{A Taste of Casbah's ORM}
    \lstinputlisting{code/casbah_mapping.scala}
\end{codeframe}

\subsection{STM + MongoDB via Akka}
\begin{itemizecodeframe}
    \frametitle{STM + MongoDB via Akka}
    \item Akka has an implementation of STM inspired by Clojure's; allows datastructures such as Maps and Vectors to become transactional.
    \item Akka STM supports persistence to several backends including MongoDB.
    \item Allows you to setup relatively simple, code managed concurrent transactions with state stored safely in MongoDB.
    \item MongoDB persistence based around Casbah as of Akka 1.0 (first milestone released today)
    \item Supports JTA; not yet distributed (Dependent on Multiverse, which is working on distributed STM)
\end{itemizecodeframe}


\section{Closing}
\begin{itemizeframe}
    \frametitle{Links}
    \item mongo-scala-driver {\scriptsize http://github.com/alaz/mongo-scala-driver}
    \item lift-mongo {\scriptsize http://www.assembla.com/wiki/show/liftweb/MongoDB}
    \item FourSquare's Lift Mongo DSL Code \ldots coming soon? {\scriptsize @jliszka}
    \item Casbah {\scriptsize http://novus.github.com/docs/casbah} (For current release - moving soon.) |  Casbah Mailing List {\scriptsize http://groups.google.com/group/mongodb-casbah-user}
    \item Jorge Ortiz' ({\scriptsize @jorgeortiz85}) Libraries
        \begin{itemize}
            \item scala-javautils (Scala 2.7.x) {\scriptsize http://github.com/jorgeortiz85/scala-javautils}
            \item scalaj-collection (Scala 2.8.x) {\scriptsize http://github.com/scalaj/scalaj-collection}
        \end{itemize}
    \item Recommended books...
        \begin{itemize}
            \item Programming Scala (Subramaniam, Pragmatic Bookshelf, 2009)
            \item Programming Scala (Payne \& Wampler, O'Reilly 2009)
            \item Coming soon ...``MongoDB: A Quick Start Guide'' from The Pragmatic Bookshelf (by me).
        \end{itemize}
\end{itemizeframe}


\begin{itemizeframe}
    \frametitle{Contact Info}
    \item Twitter: {\scriptsize @rit} | Email: {\scriptsize brendan@10gen.com} | Github: {\scriptsize http://github.com/bwmcadams}
    \item IRC - freenode.net \#mongodb | MongoDB Mailing List {\scriptsize http://groups.google.com/group/mongodb-user} | Twitter: {\scriptsize @mongodb}

    \item Hadoop + MongoDB Integration Webinar 11/15 - see \url{http://10gen.com} for details.
    \item Commercial Support, Training and more from 10gen \url{http://10gen.com}
        \begin{itemize}
            \item 10gen is hiring! We're looking for smart engineers in New York \& San Francisco
            \item MongoDB For Administrators Training in Chicago, 11/16 - 11/17 \url{http://www.10gen.com/training}
        \end{itemize}
    
\end{itemizeframe}

\end{document}


