\documentclass{beamer}

% Copyright 2010 by Brendan W. McAdams <bwmcadams@gmail.com>
%
% You may redistribute the content of this presentation for your own needs 
% provided you give credit to its author. In other words: "Don't be a dick."
% % Latex code based upon the "Generic Presentation 15-45 minutes" template from Beamer.


\mode<presentation>
{
  % \usetheme{Dresden}      
  \usetheme{CambridgeUS}      
  \usecolortheme{seahorse}
  \setbeamercovered{transparent}
}


\usepackage{hyperref}
\usepackage{xcolor}
\usepackage{listings}

\hypersetup{backref,  
pdfpagemode=FullScreen,  
colorlinks=false}

 % "define" Scala
\lstdefinelanguage{Scala}{
   morekeywords={abstract,case,catch,class,def,%
     do,else,extends,false,final,finally,%
     for,forSome,if,implicit,import,lazy,%
     match,mixin,%
     new,null,object,override,package,%
     private,protected,requires,return,sealed,%
     super,this,throw,trait,true,try,%
     type,val,var,while,with,yield},
   otherkeywords={=>,<-,<\%,<:,>:,\#,@},
   sensitive=true,
   morecomment=[l]{//},
   morecomment=[n]{/*}{*/},
   morestring=[b]",
   morestring=[b]',
   morestring=[b]"""
 }


\lstdefinelanguage{JavaScript}{
    keywords={typeof, new, true, false, catch, function, return, null, catch, switch, var, if, in, while, do, else, case, break}
    keywordstyle=\color{blue}\bfseries,
    ndkeywords={class, export, boolean, throw, implements, import, this},
    ndkeywordstyle=\color{darkgray}\bfseries,
    identifierstyle=\color{black},
    sensitive=false,
    comment=[l]{//},
    morecomment=[s]{/*}{*/},
    commentstyle=\color{purple}\ttfamily,
    stringstyle=\color{red}\ttfamily,
    morestring=[b]',
    morestring=[b]"
}
\usepackage{caption}
\DeclareCaptionFont{white}{\color{white}}
\DeclareCaptionFormat{listing}{\colorbox{black}{\parbox{\textwidth}{#1#2#3}}}
\captionsetup[lstlisting]{format=listing,labelfont=white,textfont=white}


\usepackage{color}
\definecolor{dkgreen}{rgb}{0,0.6,0}
\definecolor{gray}{rgb}{0.5,0.5,0.5}
\definecolor{mauve}{rgb}{0.58,0,0.82}
\definecolor{lightgray}{rgb}{.9,.9,.9}
\definecolor{darkgray}{rgb}{.4,.4,.4}
\definecolor{purple}{rgb}{0.65, 0.12, 0.82}

% Default settings for code listings
\lstset{
  frame=tb,%
  language=Scala,%
  aboveskip=3mm,%
  belowskip=3mm,%
  xleftmargin=4mm,%
  showstringspaces=false,%
  keepspaces=true,%
  columns=[c]fixed,%
  basicstyle={\tiny\ttfamily},%
  escapechar=¤,%
  numbers=none,%
  numberstyle=\tiny\color{gray},%
  keywordstyle=\color{blue},%
  commentstyle=\color{dkgreen},%
  stringstyle=\color{mauve},%
  frame=single,%
  breaklines=true,%
  breakatwhitespace=true,%
  tabsize=2,%
  firstnumber=0,%
  numbersep=1.5mm,%
  numberstyle=\tiny%
}

% \lstdefinestyle{floating}{%
%     xleftmargin=10pt,%
%     xrightmargin=5pt,%
%     aboveskip=4mm,%
%     belowskip=4mm,%
%     fontadjust=true,%
%     columns=[c]flexible,%
%     keepspaces=true,%
%     basewidth={0.5em, 0.425em},%
%     tabsize=2,%
%     basicstyle=\renewcommand{\baselinestretch}{0.95}\ttfamily,%
%     commentstyle=\rm,%
%     keywordstyle=\bfseries,%
%     mathescape=true,%
%     captionpos=b,%
%     framerule=0.3pt,%
%     firstnumber=0,%
%     numbersep=1.5mm,%
%     numberstyle=\tiny,%
%     float=tbp,%
%     frame=tblr,%
%     framesep=5pt,%
%     framexleftmargin=3pt,%
%     abovecaptionskip=\smallskipamount,%
%     belowcaptionskip=\smallskipamount,%
% } % to define: caption, label

\newcommand{\code}[1]{%
    \lstinline[%keywordstyle=,%
               flexiblecolumns=true,%
               basicstyle=\ttfamily]£#1£}

\newenvironment{itemizeframe}
               {\begin{frame}\startitemizeframe} 
               {\stopitemizeframe\end{frame}}
              
\newenvironment{codeframe}
                {\begin{frame}[allowframebreaks,allowdisplaybreaks]}
                {\end{frame}}
                
\newenvironment{itemizecodeframe}
              {\begin{frame}[allowframebreaks,allowdisplaybreaks]
              \startitemizeframe} 
              {\stopitemizeframe\end{frame}}

\newcommand\startitemizeframe{\begin{itemize}} \newcommand\stopitemizeframe{\end{itemize}}

\usepackage[english]{babel}
% or whatever

\usepackage[latin1]{inputenc}
% or whatever

\usepackage{times}
\usepackage[T1]{fontenc}
% Or whatever. Note that the encoding and the font should match. If T1
% does not look nice, try deleting the line with the fontenc.


\title[Integrating Scala + MongoDB]{Scala with MongoDB} % (optional, use only with long paper titles)

%\subtitle
%{Presentation Subtitle} % (optional)

\author[B.W. McAdams]{Brendan W. McAdams}
% - Use the \inst{?} command only if the authors have different
%   affiliation.

\institute[Novus Partners]{Novus Partners, Inc.}
% - Use the \inst command only if there are several affiliations.
% - Keep it simple, no one is interested in your street address.

\date[NY Scala Enthusiasts - 8/8/10]
     {New York Scala Enthusiasts\\ Aug. 8, 2010}

\subject{Using Scala with MongoDB}
% This is only inserted into the PDF information catalog. Can be left
% out. 



% If you have a file called "university-logo-filename.xxx", where xxx
% is a graphic format that can be processed by latex or pdflatex,
% resp., then you can add a logo as follows:
\logo{\includegraphics[width=1.25cm]{novus-logo}}





% If you wish to uncover everything in a step-wise fashion, uncomment
% the following command: 

%\beamerdefaultoverlayspecification{<+->}


\begin{document}


\section{Interlude: Helping Scala + Java play nice together.}
\subsection{Java <-> Scala Basics}
\begin{itemizeframe}
    \frametitle{Helping Java + Scala Interact}
    \item Implicits, ``Pimp My Library'' and various conversion helper tools simplify the work of interacting with Java.
    \item<2-> Scala and Java have their own completely different collection libraries. 
    \item<3-> Some builtins ship with Scala to make this easier.
\end{itemizeframe}


\begin{itemizeframe}
    \frametitle{Interoperability in Scala 2.7.x}
    \item Scala 2.7.x shipped with \texttt{scala.collection.jcl}.
    \item<2-> \texttt{scala.collection.jcl.Conversions} contained some implicit converters, but only to and from the wrapper versions - no support for ``real'' Scala collections.
    \item<3-> Neglected useful base interfaces like \texttt{Iterator} and \texttt{Iterable}
    \item<3-> @jorgeortiz85 provided \texttt{scala-javautils}, which used ``Pimp My Library'' to do a better job.
\end{itemizeframe}

\begin{itemizeframe}
    \frametitle{Interoperability in Scala 2.8.x}
    \item Scala 2.8.x improves the interop game significantly.
    \item<2-> JCL is gone - focus has shifted to proper interoperability w/ built-in types.
    \item<3-> \texttt{scala.collection.jcl.Conversions} replaced by \texttt{scala.collection.JavaConversions} - provides implicit conversions to \& from Scala \& Java Collections.
    \item<3-> Includes support for the things missing in 2.7 (\texttt{Iterable}, \texttt{Iterator}, etc.)
    \item<3-> Great for places where the compiler can guess what you want (implicits); falls short in some cases (like BSON Encoding, as we found in Casbah)
    \item<4-> @jorgeortiz85 has updated \texttt{scala-javautils} for 2.8 with  \texttt{scalaj-collection}
    \item<4-> Explicit \texttt{asJava} / \texttt{asScala} methods for conversions.  Adds \texttt{foreach} method to Java collections.
\end{itemizecodeframe}

\subsection{Implicits and Pimp Hats}
\begin{itemizeframe}
    \frametitle{So WTF is an `Implicit', anyway?}
    \item Implicit Arguments
        \begin{itemize}
        \item `Explicit' arguments indicates a method argument you pass, well \emph{explicitly}.
        \item `Implicit' indicates a method argument which is\ldots \emph{implied}. (But you can pass them explicitly too.)
        \item Implicit arguments are passed in Scala as an additional argument list:
            \lstinputlisting{code/implicit_sample_arg1.scala}
        \pause
        \item How does this differ from default arguments?
        \end{itemize}
\end{itemizeframe}

\begin{itemizeframe}
    \frametitle{So WTF is an `Implicit', anyway?}
    \item Implicit Methods/Conversions
        \begin{itemize}
            \item If you try passing a type to a Scala method argument which  doesn't match\ldots
                \lstinputlisting{code/implicit_sample_method1.scala}
                
            \item<2-> A fast and loose example, but simple.  Fails to compile.
            \item<3-> But with implicit methods, we can provide a conversion path\ldots
                \lstinputlisting{code/implicit_sample_method2.scala}
            \item<4-> In a dynamic language, this may be called ``monkey patching''. Unlike Perl, Python, etc. Scala resolves implicits at compile time.
        \end{itemize}
\end{itemizeframe}

\begin{itemizeframe}
    \frametitle{Pimp My Library}
    \item {\tiny Coined by Martin Odersky in a 2006 Blog post.  Similar to C\# extension methods, Ruby modules.}
    \item<2-> {\scriptsize Uses implicit conversions to tack on new methods at runtime.}
    \item<3-> {\scriptsize Either return a new ``Rich\_'' or anonymous class}\ldots
        \lstinputlisting{code/pimp_sample1.scala}
    \item<3-> {\tiny A note: with regards to type boundaries, \texttt{[A <: SomeType]} won't allow implicitly converted values.  You can whitelist them by using \texttt{[A <\% SomeType]} instead.}
\end{itemizeframe}
\end{document}


