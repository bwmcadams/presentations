\documentclass[xcolor=dvipsnames]{beamer}

% Copyright 2010 by Brendan W. McAdams <brendan@10gen.com>
%
% You may redistribute the content of this presentation for your own needs 
% provided you give credit to its author. In other words: "Don't be a dick."
% % Latex code based upon the "Generic Presentation 15-45 minutes" template from Beamer.


\mode<presentation>
{
  % \usetheme{Dresden}      
  \usetheme{Madrid}
  \setbeamercolor{structure}{fg=OliveGreen!125!black} 
        
  % \usecolortheme{dolphin}
  \setbeamercovered{transparent}
}


\usepackage{hyperref}
\usepackage{xcolor}
\usepackage{listings}

\hypersetup{backref,  
pdfpagemode=FullScreen,  
colorlinks=false}

 % "define" Scala
\lstdefinelanguage{Scala}{
   morekeywords={abstract,case,catch,class,def,%
     do,else,extends,false,final,finally,%
     for,forSome,if,implicit,import,lazy,%
     match,mixin,%
     new,null,object,override,package,%
     private,protected,requires,return,sealed,%
     super,this,throw,trait,true,try,%
     type,val,var,while,with,yield},
   otherkeywords={=>,<-,<\%,<:,>:,\#,@},
   sensitive=true,
   morecomment=[l]{//},
   morecomment=[n]{/*}{*/},
   morestring=[b]",
   morestring=[b]',
   morestring=[b]"""
 }


\lstdefinelanguage{JavaScript}{
    keywords={typeof, new, true, false, catch, function, return, null, catch, switch, var, if, in, while, do, else, case, break}
    keywordstyle=\color{blue}\bfseries,
    ndkeywords={class, export, boolean, throw, implements, import, this},
    ndkeywordstyle=\color{darkgray}\bfseries,
    identifierstyle=\color{black},
    sensitive=false,
    comment=[l]{//},
    morecomment=[s]{/*}{*/},
    commentstyle=\color{purple}\ttfamily,
    stringstyle=\color{red}\ttfamily,
    morestring=[b]',
    morestring=[b]"
}
\usepackage{caption}
\DeclareCaptionFont{white}{\color{white}}
\DeclareCaptionFormat{listing}{\colorbox{black}{\parbox{\textwidth}{#1#2#3}}}
\captionsetup[lstlisting]{format=listing,labelfont=white,textfont=white}


\usepackage{color}
\definecolor{dkgreen}{rgb}{0,0.6,0}
\definecolor{gray}{rgb}{0.5,0.5,0.5}
\definecolor{mauve}{rgb}{0.58,0,0.82}
\definecolor{lightgray}{rgb}{.9,.9,.9}
\definecolor{darkgray}{rgb}{.4,.4,.4}
\definecolor{purple}{rgb}{0.65, 0.12, 0.82}

% Default settings for code listings
\lstset{
  frame=tb,%
  language=JavaScript,%
  aboveskip=1.5mm,%
  belowskip=3.5mm,%
  xleftmargin=1mm,%
  xrightmargin=2.5mm,%
  showstringspaces=false,%
  keepspaces=true,%
  columns=[c]fixed,%
  basicstyle={\scriptsize\ttfamily},%
  escapechar=¤,%
  numbers=left,%
  numberstyle=\tiny\color{yellow},%
  keywordstyle=\color{blue},%
  commentstyle=\color{dkgreen},%
  stringstyle=\color{mauve},%
  frame=single,%
  breaklines=true,%
  breakatwhitespace=true,%
  tabsize=4,%
  firstnumber=0,%
  numbersep=1.5mm,%
  numberstyle=\tiny,
  title=\lstname%
}

\newcommand{\code}[1]{%
    \lstinline[%keywordstyle=,%
               flexiblecolumns=true,%
               basicstyle=\ttfamily]_#1_}

\newenvironment{itemizeframe}
               {\begin{frame}\startitemizeframe} 
               {\stopitemizeframe\end{frame}}
              
\newenvironment{codeframe}
                {\begin{frame}[allowframebreaks,allowdisplaybreaks]}
                {\end{frame}}
                
\newenvironment{itemizecodeframe}
              {\begin{frame}[allowframebreaks,allowdisplaybreaks]
              \startitemizeframe} 
              {\stopitemizeframe\end{frame}}

\newcommand\startitemizeframe{\begin{itemize}} \newcommand\stopitemizeframe{\end{itemize}}

\usepackage[english]{babel}
% or whatever

\usepackage[latin1]{inputenc}
% or whatever

\usepackage{times}
\usepackage[T1]{fontenc}
% Or whatever. Note that the encoding and the font should match. If T1
% does not look nice, try deleting the line with the fontenc.


\title{MongoDB Plugin \& Toolchain for Hadoop} % (optional, use only with long paper titles)

\institute[10gen, Inc.]{10gen, Inc.}

%\subtitle
%{Presentation Subtitle} % (optional)

\author[B.W. McAdams]{Brendan W. McAdams}
% - Use the \inst{?} command only if the authors have different
%   affiliation.

\date{Dec. 3, 2010 @ MongoSV}

\subject{The Elephant In The Room: The MongoDB Plugin and Toolchain for Hadoop}
% This is only inserted into the PDF information catalog. Can be left
% out. 


\logo{\includegraphics[width=2.25cm]{images/mongodb-logo}}

% If you wish to uncover everything in a step-wise fashion, uncomment
% the following command: 

%\beamerdefaultoverlayspecification{<+->}


\begin{document}

\begin{frame}
  \titlepage
  \begin{center}
  \includegraphics{images/mongo_hadoop.png}
  \end{center}
\end{frame}

\section{Introduction}

\subsection[Mongo + Hadoop]{Why Integrate?}
\begin{itemizeframe}
    \frametitle{Hadoop Explained\ldots}
	\item Started in February 2006 as part of the Apache Lucene project
	\item Based upon Google's MapReduce and GFS Papers
	\item Allows distributed, scalable data processing of huge datasets
	\item Java based, but with support for other JVM and Non-JVM Languages
	\item Lots of ecosystem tools to simplify interaction such as Pig and Hive
	\item In use at New York Times, Last.fm, Yahoo!, Amazon, Facebook and {\em many more companies\ldots}
	\item Great tools for temporary Hadoop clusters such as the Cloudera Cluster Tools, Apache Whirr and Amazon's Elastic MapReduce.
\end{itemizeframe}

\begin{itemizeframe}
    \frametitle{Why Integrate MongoDB?}
        \item<2-> Language: JavaScript only; not everyone wants to write JavaScript for data processing.
            \begin{itemize}
                \item<3->Static Typing: JavaScript is dynamically typed which has limited value for jobs where a stronger type system is desired.
                \item<3->JVM Ecosystem: The JVM has a large number of libraries available for assisting in calculations and analysis which are available from Hadoop.
            \end{itemize}
        \item<4-> Concurrency: Current JS Implementation is limited to one JS execution per server at a time.
        \item<5-> Scalability: Not a lot of ability to scale MongoDB's MapReduce except in cases of sharding.
        \item<6-> Integration: More development shops are integrating their data analysis into Hadoop jobs---opportunity to merge data from multiple systems including MongoDB for analysis.
    \item<7->Hadoop is not a panacea---queries won't necessarily be fast(er) but it can handle larger scale and free up your MongoDB servers for long jobs.
\end{itemizeframe}

\begin{itemizeframe}
    \frametitle{Mongo + Hadoop \ldots}
    \item Integrating MongoDB and Hadoop to read \& write data from/to MongoDB via Hadoop
    \item 10gen has been working on a plugin to integrate the two systems, written in Pure Java
    \item About 6 months ago I explored the idea in Scala with a project called `Luau'
    \item Support for pure MapReduce as well as Pig (Currently output only - input coming soon)
    \item With Hadoop Streaming (soon), write your MapReduce in Python, Ruby or Perl
\end{itemizeframe}

\subsection[The Code\ldots Revealed]{MongoDB + MapReduce in Java}

\begin{codeframe}
    \frametitle{MongoDB + MapReduce in Java}
    We're going to walk through a sample application which parses US Treasury Bond historical Bid Curves since January 1990, and calculates an annual average for the 10 year Treasury.

    \begin{itemize}
        \item A sample of our dataset:
        \lstinputlisting{code/sample_treasury.js}
        \pagebreak 
        \item The MongoDB JavaScript version of our demo:
        \lstinputlisting{code/mongo_treasury_mr.js}
        \pagebreak
        \item The same work can be done in Hadoop\ldots
    \end{itemize}
    
\end{codeframe}

\begin{codeframe}
    \frametitle{MongoDB Hadoop Job for Treasury Data}
    \framesubtitle{Configuration}
    The Configuration is specified in XML
    \lstinputlisting[language=xml]{code/mongo-treasury_yield.xml}
\end{codeframe}

\begin{codeframe}
    \frametitle{MongoDB Hadoop Job for Treasury Data}
    \framesubtitle{Configuration}
    The Job is loaded \& configured via a Java Tool:
    \lstinputlisting[language=Java]{code/TreasuryYieldXMLConfig.java}
\end{codeframe}

\begin{codeframe}
    \frametitle{MongoDB Hadoop Job for Treasury Data}
    \framesubtitle{Configuration}
    MongoTool is a helper utility for doing XML configs with queries, etc:
    \lstinputlisting[language=JavaScript]{code/MongoTool.java}
\end{codeframe}

\begin{codeframe}
    \frametitle{MongoDB Hadoop Job for Treasury Data}
    \framesubtitle{Mapper}
    The Map function:
    \lstinputlisting[language=Java]{code/TreasuryYieldMapper.java}
\end{codeframe}
\begin{codeframe}
    \frametitle{MongoDB Hadoop Job for Treasury Data}
    \framesubtitle{Reducer}
    The Reduce function: 
    \lstinputlisting[language=Java]{code/TreasuryYieldReducer.java}
\end{codeframe}

\begin{itemizeframe}
    \frametitle{MongoDB Hadoop Job For Treasury Data}
    \framesubtitle{Running}
    \item Make sure the Java Driver is in the `lib' directory on every cluster member, e.g. \code{sudo cp lib/mongo-java-driver-2.3.jar /usr/lib/hadoop-0.20/lib/} (Bounce hadoop after adding libraries)
    \item Build the Hadoop Jar (\code{ant resolve; ant jar})
    \item Run the Job!
    
\end{itemizeframe}

\subsection{Pig for ETL}
\begin{itemizecodeframe}
    \frametitle{A Modified Pig Tutorial with MongoDB Output}
    \item We took the stock Pig Tutorial and modified it to save to MongoDB
    \item Input file is an anonymized Search Engine Log:
        \lstinputlisting{code/excite_log_snippet.txt}
    \item The Script is only lightly modified from stock pig:
        \lstinputlisting{code/pigtutorial.pig}
    
\end{itemizecodeframe}


\begin{itemizeframe}
    \frametitle{Mongo Hadoop Plugin}
    \item Starting today, you can download and use the Hadoop Plugin:  \url{https://github.com/mongodb/mongo-hadoop}
    \item Still an Alpha---{\em not} production ready.
    \item Continuing Development\ldots
        \begin{itemize}
            \item Working on support for Hadoop Streaming, which allows Python, Ruby, and Perl to be used instead of Java (Waiting to migrate to .21 for Binary support)
            \item Pig Input support.
            \item Support for Sharding concurrency (Once one of you asks for it <hint-hint>)
            \item Exploring other complimentary toolchains like Cascading.
            \item Tell us how you are using Hadoop or how you want to integration MongoDB + Hadoop\ldots So we can focus on the areas you, the users, want and need.
        \end{itemize}
\end{itemizeframe}
\begin{itemizeframe}
    \frametitle{Questions?}
        \item Twitter: {\bf @rit} | mongodb: {\bf @mongodb} | 10gen: {\bf @10gen }
        \item email: {\bf brendan@10gen.com}	
		\item mongo-hadoop \ldots Available in a Pre-Production Alpha: \url{https://github.com/mongodb/mongo-hadoop}
		\item Report Bugs, request features: \url{https://github.com/mongodb/mongo-hadoop/issues}
    \item Pressing Questions?
    \begin{itemize}
        \item IRC - {freenode.net \bf\#mongodb}
        \item MongoDB Users List - \url{http://groups.google.com/group/mongodb-user}
    \end{itemize}
    \item {\Large 10gen is {\em hiring!}  We need smart engineers in both NY and Bay Area: \url{http://10gen.com/jobs}}
    \item Up Next: 15 Minute Break followed by Mathias Stearn on ``MongoDB Internals: The Storage Engine''
\end{itemizeframe}




\end{document}
