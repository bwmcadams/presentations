\documentclass{beamer}

% Copyright 2010 by Brendan W. McAdams <bwmcadams@gmail.com>
%
% You may redistribute the content of this presentation for your own needs 
% provided you give credit to its author. In other words: "Don't be a dick."
% % Latex code based upon the "Generic Presentation 15-45 minutes" template from Beamer.


\mode<presentation>
{
  %\usetheme{Dresden}      
  \usetheme{Boadilla}      
  %\usecolortheme{wolverine}
  \setbeamercovered{transparent}
}


\usepackage{hyperref}
\usepackage{xcolor}
\usepackage{listings}

\hypersetup{backref,  
pdfpagemode=FullScreen,  
colorlinks=false}

 % "define" Scala
\lstdefinelanguage{Scala}{
   morekeywords={abstract,case,catch,class,def,%
     do,else,extends,false,final,finally,%
     for,forSome,if,implicit,import,lazy,%
     match,mixin,%
     new,null,object,override,package,%
     private,protected,requires,return,sealed,%
     super,this,throw,trait,true,try,%
     type,val,var,while,with,yield},
   otherkeywords={=>,<-,<\%,<:,>:,\#,@},
   sensitive=true,
   morecomment=[l]{//},
   morecomment=[n]{/*}{*/},
   morestring=[b]",
   morestring=[b]',
   morestring=[b]"""
 }


\lstdefinelanguage{JavaScript}{
    keywords={typeof, new, true, false, catch, function, return, null, catch, switch, var, if, in, while, do, else, case, break}
    keywordstyle=\color{blue}\bfseries,
    ndkeywords={class, export, boolean, throw, implements, import, this},
    ndkeywordstyle=\color{darkgray}\bfseries,
    identifierstyle=\color{black},
    sensitive=false,
    comment=[l]{//},
    morecomment=[s]{/*}{*/},
    commentstyle=\color{purple}\ttfamily,
    stringstyle=\color{red}\ttfamily,
    morestring=[b]',
    morestring=[b]"
}
\usepackage{caption}
\DeclareCaptionFont{white}{\color{white}}
\DeclareCaptionFormat{listing}{\colorbox{black}{\parbox{\textwidth}{#1#2#3}}}
\captionsetup[lstlisting]{format=listing,labelfont=white,textfont=white}


\usepackage{color}
\definecolor{dkgreen}{rgb}{0,0.6,0}
\definecolor{gray}{rgb}{0.5,0.5,0.5}
\definecolor{mauve}{rgb}{0.58,0,0.82}
\definecolor{lightgray}{rgb}{.9,.9,.9}
\definecolor{darkgray}{rgb}{.4,.4,.4}
\definecolor{purple}{rgb}{0.65, 0.12, 0.82}

% Default settings for code listings
\lstset{
  frame=tb,%
  language=Python,%
  aboveskip=1.5mm,%
  belowskip=1.5mm,%
  xleftmargin=1mm,%
  xrightmargin=1mm,%
  showstringspaces=false,%
  keepspaces=true,%
  columns=[c]fixed,%
  basicstyle={\tiny\ttfamily},%
  escapechar=¤,%
  numbers=none,%
  numberstyle=\tiny\color{yellow},%
  keywordstyle=\color{blue},%
  commentstyle=\color{dkgreen},%
  stringstyle=\color{mauve},%
  frame=single,%
  breaklines=true,%
  breakatwhitespace=true,%
  tabsize=4,%
  firstnumber=0,%
  numbersep=1.5mm,%
  numberstyle=\tiny%
}

% \lstdefinestyle{floating}{%
%     xleftmargin=10pt,%
%     xrightmargin=5pt,%
%     aboveskip=4mm,%
%     belowskip=4mm,%
%     fontadjust=true,%
%     columns=[c]flexible,%
%     keepspaces=true,%
%     basewidth={0.5em, 0.425em},%
%     tabsize=2,%
%     basicstyle=\renewcommand{\baselinestretch}{0.95}\ttfamily,%
%     commentstyle=\rm,%
%     keywordstyle=\bfseries,%
%     mathescape=true,%
%     captionpos=b,%
%     framerule=0.3pt,%
%     firstnumber=0,%
%     numbersep=1.5mm,%
%     numberstyle=\tiny,%
%     float=tbp,%
%     frame=tblr,%
%     framesep=5pt,%
%     framexleftmargin=3pt,%
%     abovecaptionskip=\smallskipamount,%
%     belowcaptionskip=\smallskipamount,%
% } % to define: caption, label

\newcommand{\code}[1]{%
    \lstinline[%keywordstyle=,%
               flexiblecolumns=true,%
               basicstyle=\ttfamily]£#1£}

\newenvironment{itemizeframe}
               {\begin{frame}\startitemizeframe} 
               {\stopitemizeframe\end{frame}}
              
\newenvironment{codeframe}
                {\begin{frame}[allowframebreaks,allowdisplaybreaks]}
                {\end{frame}}
                
\newenvironment{itemizecodeframe}
              {\begin{frame}[allowframebreaks,allowdisplaybreaks]
              \startitemizeframe} 
              {\stopitemizeframe\end{frame}}

\newcommand\startitemizeframe{\begin{itemize}} \newcommand\stopitemizeframe{\end{itemize}}

\usepackage[english]{babel}
% or whatever

\usepackage[latin1]{inputenc}
% or whatever

\usepackage{times}
\usepackage[T1]{fontenc}
% Or whatever. Note that the encoding and the font should match. If T1
% does not look nice, try deleting the line with the fontenc.


\title{The Elephant In the Room: MongoDB + Hadoop} % (optional, use only with long paper titles)

\institute[10gen]{10gen, Inc.}

%\subtitle
%{Presentation Subtitle} % (optional)

\author[B.W. McAdams]{Brendan W. McAdams}
% - Use the \inst{?} command only if the authors have different
%   affiliation.

\date{Mongo Chicago Conference - October 2010}

\subject{MongoDB + Hadoop}
% This is only inserted into the PDF information catalog. Can be left
% out. 



% If you have a file called "university-logo-filename.xxx", where xxx
% is a graphic format that can be processed by latex or pdflatex,
% resp., then you can add a logo as follows:
%\logo{\includegraphics[width=3.75cm]{sluggy-logo}}



% Delete this, if you do not want the table of contents to pop up at
% the beginning of each subsection:
% \AtBeginSubsection[]
% {
%   \begin{frame}<beamer>
%   \frametitle{Outline}
%     \tableofcontents[currentsection,currentsubsection,currentsubsubsection]
%   \end{frame}
% }


% If you wish to uncover everything in a step-wise fashion, uncomment
% the following command: 

%\beamerdefaultoverlayspecification{<+->}


\begin{document}

\begin{frame}
  \titlepage
\end{frame}

\begin{frame}
\frametitle{Outline}
  \tableofcontents
  % You might wish to add the option [pausesections]
\end{frame}

\section{Introduction}

\subsection[Mongo + Hadoop]{Why Integrate?}
\begin{itemizeframe}
    \frametitle{Hadoop Explained\ldots}
	\item Started in February 2006 as part of the Apache Lucene project
	\item Based upon Google's MapReduce and GFS Papers
	\item Allows distributed, scalable data processing of huge datasets
	\item Java based, but with support for other JVM and Non-JVM Languages
	\item Lots of ecosystem tools to simplify interaction such as Pig and Hive
	\item In use at New York Times, Last.fm, Yahoo!, Amazon, Facebook and {\em many more companies\ldots}
	\item Great tools for temporary Hadoop clusters such as the Cloudera Cluster Tools, Apache Whirr and Amazon's Elastic MapReduce.
\end{itemizeframe}

\begin{itemizeframe}
    \frametitle{Why Integrate MongoDB?}
	\item Easiest Answer: People keep asking for it \ldots
	\item Limitations in MongoDB's built in MapReduce
    \begin{itemize}
        \item<2-> Language: JavaScript only; not everyone wants to write JavaScript for data processing.
        \item<3-> Concurrency: Current JS Implementation is limited to one JS execution per server at a time.
        \item<4-> Scalability: Not a lot of ability to scale MongoDB's MapReduce except in cases of sharding.
    \end{itemize}
\end{itemizeframe}

\begin{itemizeframe}
    \frametitle{Mongo + Hadoop \ldots}
    \item Integrating MongoDB and Hadoop to read \& write data from/to MongoDB via Hadoop
    \item 10gen has been working on a plugin to integrate the two systems, written in Pure Java
    \item About 6 months ago I explored the idea in Scala with a project called `Luau'
    \item Support for pure MapReduce as well as Pig (Currently output only - input coming soon)
    \item With Hadoop Streaming (soon), write your MapReduce in Python or Ruby
\end{itemizeframe}

\subsection[The Code\ldots Revealed]{MongoDB + MapReduce in Java}
% \begin{codeframe}
%     \frametitle{Input Data}
%     \lstinputlisting[language=JavaScript]{code/wordCountIn.js}
% \end{codeframe}
\begin{codeframe}
    \frametitle{MongoDB + MapReduce in Java}
    \lstinputlisting{code/WordCount.java}
\end{codeframe}

\begin{codeframe}
    \frametitle{The Results}
    \lstinputlisting[language=JavaScript]{code/wordCountOut.js}
\end{codeframe}

\subsection[The Code\ldots Revealed]{Using Pig for High Level Data Analysis}
\begin{codeframe}
    \frametitle{Sample Input Data}
    \lstinputlisting{code/sampleData.log}
\end{codeframe}

\begin{codeframe}
    \frametitle{The Pig Script}
    \lstinputlisting{code/test.pig}
\end{codeframe}

\begin{codeframe}
    \frametitle{The Output}
    \lstinputlisting{code/pigTestOut.js}
\end{codeframe}

\begin{itemizeframe}
    \frametitle{Questions?}
    \framesubtitle{Contact Info}
    \item Contact Me
    \begin{itemize}
        \item twitter: {\bf @rit}
        \item email: {\bf brendan@10gen.com}	
		\item mongo-hadoop \ldots Available Soon in Alpha Form: \url{http://github.com/mongodb/mongo-hadoop}
    \end{itemize}
    \item Pressing Questions?
    \begin{itemize}
        \item IRC - {freenode.net \bf\#mongodb}
        \item MongoDB Users List - \url{http://groups.google.com/group/mongodb-user}
    \end{itemize}
\end{itemizeframe}




\end{document}
